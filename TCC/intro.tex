\section{Introdução}
\label{sec:intro}

O \textit{E-commerce}, ou comércio eletrônico, pode ser definido como uma modalidade de negócio onde todo o processo de compra é feito exclusivamente online, isto é, em aplicativos móveis e computadores é realizado desde a escolha do produto ao pagamento. O comércio eletrônico  como conhecemos hoje iniciou no final da década de 1960, mas desde 1993, novas tecnologias permitem as empresas realizar  funções de negócios eletrônicos (e-business) com maior eficiência, rapidez e  menores custos. Dados mais recentes da Associação Brasileira de Comércio  Eletrônico (ABComm) estimam que, em 2020, 20,2 milhões de consumidores realizaram uma compra on-line pela primeira vez e 150 mil lojas começaram a  vender por meio das plataformas digitais. Foram mais de 342 milhões de  negociações feitas pela internet, com um valor médio de R$\$ 310$ e um volume financeiro estimado de R$\$ 106$ bilhões \citep{Abcomm2020}.

Até pouco tempo atrás as lojas virtuais eram utilizadas como plataforma  de ampliação de vendas de lojas físicas já renomadas, à exemplo de Casas  Bahia, Magazine Luiza, Lojas Americanas e etc., mas com a crescente do e-commerce no Brasil nos anos de 2020 e 2021, proveniente da crise do  coronavírus, observa-se que as empresas estão apostando em lojas virtuais para  a exposição e vendas dos seus produtos, dispensando, muitas vezes, o comercio  em lojas físicas.

Por se tratar de um mercado de acesso nacional, com maior eficiência, rapidez e menores custos, a ocorrência das vendas on-line é elevada. Logo, para uma  empresa se manter competitiva e sólida neste mercado é preciso planejamento,  inovação e, principalmente, entender sobre as necessidades dos clientes e como  fidelizá-los. De acordo com \cite{Efagundes2021} um empreendimento de sucesso é aquele que  consegue utilizar a tecnologia existente, adequada aos consumidores do seu  nicho de mercado. Por isso mesmo, é  fundamental conhecer o comportamento dos consumidores e as tendencias do negócio.

No lado de quem consome, pode-se citar como vantagens do e-commerce: comodidade para os  clientes, acesso a opinião de outros usuários/compradores, funcionamento em tempo integral, diversidade de produtos, variedade de opções de pagamentos,  privacidade ao usuário/comprador, cupons de desconto e etc., e são justamente  estas vantagens e diversidades de informações de dados na compra e venda de  produtos que permitem identificar os nichos de mercados de cada seguimento.

Ainda no que concerne o consumidor, após a efetivação da compra, é solicitado uma avaliação do item. Ao se avaliar com pouca precisão, sem muito critério, ou sem muito zelo, algumas situações podem ser encotradas. A primeira delas é a condição não pouco incomum de uma nota desalinhada com o comentário avaliativo, por exemplo: é possível encontrar notas objetivas elevadas (escala 1 a 5, uma nota 5), porém com comentários que demonstram insatisfação com o produto. A segunda situação diz respeito às avaliações intermediárias, no que caracteriza o que é conhecido como \textit{j-shape rating}. Os usuários plenamente satisfeitos e/ou insatisfeitos tendem a avaliar o produto corretamente, todavia, o restante deles muitas vezes acabam avaliando sem muita convicção e de forma errônea.

Destarte, vista as condições de imprecisão ao se obter essas métricas em específico, com este trabalho objetivou analisar dados de diferentes \textit{e-commerces} no Brasil, mais especificamente as notas e comentários avaliativos dos produtos com intuito de obter uma classificação pertinente da satisfação do cliente e também um tradeoff entre assertividade e custo de processamento do modelo. Nesse processo foram concebidos modelos com cinco métodos tradicionais do aprendizado de máquina utilizados, sendo eles a Regressão logística, \textit{Naive Bayes}, \textit{XGBoost}, Florestas Aleatórias, \textit{LightGBM} e, para fins comparativos analisou-se também um modelo construído com rede neural artificial \textit{long short term memory} (LSTM).

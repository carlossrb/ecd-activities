\subsubsection{Árvore de decisão}

Antes de discorrer sobre os métodos a seguir de aprendizado de máquina, faz-se necessário deixar elucidado como funcionam as árvores de decisão.

Uma árvore de decisão é um modelo de aprendizado de máquina que usa uma estrutura em forma de árvore para representar uma série de decisões e suas possíveis consequências. Cada nó interno da árvore representa uma decisão que leva a um ou mais ramos subsequentes, enquanto cada folha representa o resultado final de uma série de decisões.

As árvores de decisão são amplamente utilizadas em áreas como ciência de dados, reconhecimento de padrões, bioinformática e finanças. Elas são populares porque são fáceis de entender e interpretar, permitindo que os usuários examinem os fatores que afetam as decisões e identifiquem os resultados mais prováveis.

Para construir uma árvore de decisão, é necessário um conjunto de dados de treinamento que contenha exemplos de entrada e saída. A árvore é construída a partir da análise dos dados de treinamento para determinar as variáveis mais importantes e suas relações com as saídas.

Existem várias técnicas para a construção de árvores de decisão, incluindo o algoritmo ID3 (\textit{Iterative Dichotomiser 3}), o algoritmo C4.5 e o algoritmo CART (\textit{Classification and Regression Trees}). Cada algoritmo tem suas próprias vantagens e desvantagens, dependendo do problema em questão e dos dados disponíveis.

As árvores de decisão também podem ser usadas em conjunto com outras técnicas de aprendizado de máquina, como o \textit{ensemble learning}, para melhorar a precisão dos modelos e reduzir o risco de \textit{overfitting}.
\subsubsection{Florestas aleatórias}

De acordo com \cite{breiman2001random}, as florestas aleatórias são uma técnica de aprendizado de máquina que combina várias árvores de decisão para construir um modelo de classificação ou regressão. Cada árvore de decisão é construída a partir de um subconjunto aleatório dos dados de treinamento e um subconjunto aleatório dos recursos (também conhecidos como características ou atributos). Esses subconjuntos são criados para garantir que cada árvore de decisão seja diferente e que a floresta aleatória possa capturar várias relações entre os dados e recursos.

A construção de uma árvore de decisão é feita por meio de uma série de etapas. Inicialmente, a árvore começa com um único nó que representa todo o conjunto de dados de treinamento. Em seguida, a árvore é dividida em nós menores usando uma função de divisão que escolhe um recurso e um ponto de divisão que minimiza a impureza dos dados. A impureza é uma medida da desorganização dos dados, que pode ser medida por diferentes critérios, como a entropia ou o índice Gini. O processo de divisão é repetido recursivamente até que os nós finais sejam puros ou um critério de parada seja atingido, como uma profundidade máxima da árvore \citep{cutler2001random}.

Durante a fase de teste, a floresta aleatória retorna a classe mais comum ou a média das saídas das árvores individuais, dependendo se o problema é de classificação ou regressão, respectivamente.

As florestas aleatórias apresentam várias vantagens em relação a outras técnicas de aprendizado de máquina. Em primeiro lugar, elas têm um bom desempenho em dados de alta dimensão, onde o número de recursos é grande em relação ao número de amostras. Em segundo lugar, elas são relativamente insensíveis a outliers e dados ausentes. Em terceiro lugar, elas são facilmente paralelizáveis, permitindo que grandes conjuntos de dados sejam processados em paralelo em clusters de computadores \citep{cutler2001random}.

Em termos de aplicação, elas são amplamente utilizadas em uma variedade de problemas, como reconhecimento de padrões em imagens e sinais, detecção de fraudes em transações financeiras, análise de sentimentos em redes sociais, previsão de preços de ações e análise de dados genômicos \citep{breiman2001random}.

As equações relacionadas às florestas aleatórias são principalmente as usadas na construção de cada árvore de decisão. Por exemplo, as equações para o cálculo da impureza dos dados, que é um dos principais critérios de divisão de nós em uma árvore de decisão são dadas:

\begin{itemize}
    \item O índice Gini:
          \begin{equation}
              G_i = \sum_{k=1}^{K} p_{i,k} (1-p_{i,k}),
          \end{equation}

          onde $K$ é o número de classes, $p_{i,k}$ é a proporção de observações da classe $k$ no nó $i$.
    \item A entropia:
          \begin{equation}
              H_i = -\sum_{k=1}^{K} p_{i,k} \log(p_{i,k}),
          \end{equation}

          onde $K$ é o número de classes, $p_{i,k}$ é a proporção de observações da classe $k$ no nó $i$.

    \item O critério de divisão de Gini é dado por:
          \begin{equation}
              G_{d} = \sum_{i=1}^{q}\frac{n_i}{n}G_i,
          \end{equation}

          onde $q$ é o número de nós filhos resultantes da divisão, $n_i$ é o número de observações no nó $i$ e $n$ é o número total de observações.

    \item O critério de divisão de entropia é dado por:
          \begin{equation}
              H_{d} = -\sum_{i=1}^{q}\frac{n_i}{n}H_i,
          \end{equation}

          onde $q$ é o número de nós filhos resultantes da divisão, $n_i$ é o número de observações no nó $i$ e $n$ é o número total de observações.

\end{itemize}

Essas equações são usadas para calcular a impureza dos dados em cada nó da árvore de decisão e, assim, decidir qual recurso e ponto de divisão usar para dividir o nó em dois filhos. O processo de divisão é repetido recursivamente para construir a árvore de decisão completa. Em seguida, várias árvores de decisão são combinadas para formar a floresta aleatória.



\usetikzlibrary{fit,shapes.arrows,positioning}

\tikzset{marrow/.style={midway,red,sloped,fill, minimum height=3cm, single arrow, single arrow
            head extend=.5cm, single arrow head indent=.25cm,xscale=0.3,yscale=0.15,
            allow upside down}}
\begin{figure}[H]
    \centering
    \scalebox{0.48}{
        \begin{forest}
            for tree={l sep=3em, s sep=3em, anchor=center, inner sep=0.7em, fill=blue!50,
            circle, font=\Large\sffamily,where level=1{no edge}{}}
            [Dados de treinamento, draw, rectangle, rounded corners, orange, text=white,alias=TD
            [,red!70,alias=a1[[,alias=a2][]][,red!70,edge label={node[above=1ex,marrow]{}}[[][]][,red!70,edge label={node[above=1ex,marrow]{}}[,red!70,edge label={node[below=1ex,marrow]{}}][,alias=a3]]]]
            [,red!70,alias=b1[,red!70,edge label={node[below=1ex,marrow]{}}[[,alias=b2][]][,red!70,edge label={node[above=1ex,marrow]{}}]][[][[][,alias=b3]]]]
            [~$\cdots$~,scale=4,no edge,fill=none,yshift=-1em]
            [,red!70,alias=c1[[,alias=c2][]][,red!70,edge label={node[above=1ex,marrow]{}}[,red!70,edge label={node[above=1ex,marrow]{}}[,alias=c3][,red!70,edge label={node[above=1ex,marrow]{}}]][,alias=c4]]]
            ]
            \node[draw,fit=(a1)(a2)(a3)](f1){};
            \node[draw,fit=(b1)(b2)(b3)](f2){};
            \node[draw,fit=(c1)(c2)(c3)(c4)](f3){};
            \path (f1.south west)--(f3.south east) node[midway,below=4em] (D) {Média};
            \node[below=2em of D] (pred){Predição};
            \foreach \X in {1,2,3}{\draw[-stealth] (TD) -- (f\X.north);
                    \draw[-stealth] (f\X.south) -- (D);}
            \draw[-stealth] (D) -- (pred);
        \end{forest}
    }
    \caption{Esquema de funcionamento de uma floresta aleatória}
    \label{plt:randomForest}
\end{figure}










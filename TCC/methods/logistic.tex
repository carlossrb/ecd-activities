\subsubsection{Regressão Logística}

A regressão logística é um modelo estatístico robusto e eficiente, que permite a previsão da probabilidade de um evento binário de forma precisa e confiável \citep{hosmer2013applied}. Este tipo de evento é aquele que pode ocorrer ou não, ou que pode ser classificado em duas categorias distintas. Por exemplo, em uma análise de crédito, o cliente pode ser aprovado ou não. Na medicina, o evento pode ser a cura ou não de uma doença.

Ela apresenta um modelo linear generalizado que utiliza a função logística para modelar a relação entre as variáveis independentes e a variável dependente \citep{kleinbaum2010logistic}. A função logística é uma função sigmoide que transforma uma variável linear em uma probabilidade. A equação da função logística é dada por:

\begin{equation}
    p = \frac{1}{1 + e^{-x}}
\end{equation}

Sendo $p$ a probabilidade do evento ocorrer, $x$ uma variável linear que representa a combinação linear das variáveis independentes, e $e$ a constante de Euler.

A regressão logística utiliza a técnica de máxima verossimilhança para estimar os parâmetros do modelo a partir dos dados observados. A função de verossimilhança é maximizada para encontrar os valores dos coeficientes que melhor se ajustam aos dados. O modelo é ajustado para minimizar a diferença entre as probabilidades previstas pelo modelo e as probabilidades observadas nos dados \citep{mccullagh1989generalized}.

A regressão logística tem sido amplamente utilizada em diversas áreas, como na análise de dados de sobrevivência, na análise de dados de saúde, na análise de dados financeiros, etc. Por exemplo, é utilizada na análise de dados de sobrevivência para modelar a probabilidade de um paciente sobreviver a uma doença com base em fatores como idade, sexo, nível de educação, etc. Na análise de dados financeiros, é utilizada para modelar a probabilidade de um cliente pagar ou não uma dívida com base em fatores como histórico de crédito, renda, etc.

Na \autoref{plt:logistic} tem-se a curva sigmoide que é uma representação visual de como a probabilidade de uma variável dependente binária (como ``sim" ou ``não") muda em relação a uma variável independente contínua. Ela é uma função em forma de ``S" que começa com uma probabilidade baixa para valores muito baixos de $x$, aumenta rapidamente à medida que $x$ aumenta e se estabiliza em uma probabilidade alta para valores muito altos de $x$. A inclinação da curva sigmoide no ponto médio (onde $p = 0,5$) é o valor da inclinação da reta da regressão logística.


\begin{figure}[H]
    \centering
    \begin{tikzpicture}
        \begin{axis}[    axis lines=left,    xlabel=$x$,    ylabel=$y$,    ymin=0, ymax=1,    ytick={0,1},    xmin=-5, xmax=5,    legend pos=north west]
            \addplot[domain=-5:5, samples=100, blue, thick] {1/(1+exp(-x))};
            \addlegendentry{$\frac{1}{1+e^{-x}}$};
            \draw[dashed] (axis cs:-5,0.5) -- (axis cs:5,0.5);
            \addplot[only marks, mark=*, red] coordinates {
                    (-4, 0.1)
                    (-1, 0.3)
                    (0, 0.6)
                    (1, 0.8)
                    (2, 0.9)
                    (4, 0.96)
                };
            \addlegendentry{Dados}
        \end{axis}
    \end{tikzpicture}
    \caption{Modelo gráfico da regressão logística (curva sigmoide)}
    \label{plt:logistic}
\end{figure}
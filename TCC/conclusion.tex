\section{Conclusões}
\label{sec:conclu}

Com base na análise dos resultados, podemos concluir que a rede neural artificial LSTM apresentou a melhor performance em relação aos outros algoritmos de \textit{machine learning} avaliados. A acurácia da LSTM foi de $90\%$, o que indica que o modelo foi capaz de classificar corretamente a grande maioria dos dados.

Ao avaliar a curva ROC, podemos observar que a LSTM também apresentou a maior área sob a curva, indicando uma maior capacidade de distinguir entre as classes. Esse resultado é consistente com a análise das matrizes de confusão, que mostraram que a LSTM teve o melhor desempenho na classificação correta das amostras.

O XGBoost também apresentou resultados promissores, com uma acurácia de $82\%$ e uma área sob a curva ROC de 0,89. No entanto, as matrizes de confusão indicaram que o modelo cometeu mais erros do que a LSTM na classificação de algumas amostras.

Por outro lado, a regressão logística e o Naive Bayes apresentaram as piores performances, com acurácias de $73\%$ e $61\%$, respectivamente, e áreas sob a curva ROC menores do que os outros modelos avaliados.

Se tratando de \textit{tradeoff}, na prática, muitas vezes é necessário encontrar um equilíbrio entre a rapidez da resposta e a precisão do modelo. Isso ocorre porque muitas aplicações em tempo real exigem que a resposta seja obtida em um curto espaço de tempo, enquanto outras aplicações, como pesquisa médica ou financeira, priorizam a precisão.

A escolha do modelo ideal depende de vários fatores, como o tamanho dos dados, a complexidade do problema e a disponibilidade de recursos de computação. Modelos mais complexos, como as redes neurais, podem ter uma acurácia muito alta, mas exigem uma grande quantidade de tempo de processamento, enquanto modelos mais simples, como regressão logística ou Naive Bayes, têm tempos de processamento menores, mas podem ter uma acurácia menor.

As Florestas Aleatórias e o XGBoost são modelos intermediários em termos de complexidade e tempo de execução. Esses modelos podem ser mais adequados para muitas aplicações, pois têm uma acurácia razoável e um tempo de execução curto o suficiente para muitas tarefas em tempo real.

No caso específico apresentado no trabalho, a regressão logística tem um tempo de execução muito maior do que os outros modelos, o que pode ser um fator limitante em muitas aplicações. Por outro lado, o Naive Bayes tem um tempo de execução muito curto, mas também tem a menor acurácia entre os modelos apresentados.

A rede neural LSTM tem uma acurácia muito alta, mas um tempo de execução extremamente longo, o que pode torná-la impraticável para muitas aplicações em tempo real. No entanto, para aplicações onde a precisão é o fator mais importante, como diagnóstico médico ou detecção de fraudes financeiras, esse modelo pode ser a melhor opção.

Outrossim, é necessário destacar que o XGBoost possui algumas vantagens em relação a redes neurais como a LSTM.

Uma das delas é a sua capacidade de lidar com dados heterogêneos e faltantes de forma eficiente, o que pode ser um desafio para redes neurais. Além disso, com a visão de um negócio, onde a velocidade é importante, o XGBoost é um modelo mais simples e rápido de treinar do que redes neurais, o que pode ser uma vantagem em cenários em que o tempo e recursos computacionais são limitados.

O XGBoost também apresenta interpretabilidade, uma vez que ele permite identificar as variáveis mais importantes para a classificação dos dados. As redes neurais, embora apresentem alta performance em muitas tarefas, são frequentemente consideradas caixas pretas, dificultando a interpretação dos resultados.

Portanto, embora a LSTM tenha apresentado a melhor performance na análise classificatória realizada, o XGBoost pode ser uma boa opção em cenários em que a interpretabilidade e o processamento de dados faltantes são prioridades.

Por outro lado, a LSTM consegue lidar com dados sequenciais e com dependências de longo prazo, o que pode ser um desafio para algoritmos de aprendizado de máquina tradicionais. Isso torna a LSTM uma escolha popular para tarefas de processamento de linguagem natural, reconhecimento de fala, análise de séries temporais, entre outras aplicações em que o histórico de dados é importante. Além disso, ela é capaz de aprender padrões complexos em dados sequenciais sem a necessidade de engenharia manual de características, o que pode ser um processo demorado e sujeito a erros nos outros algoritmos, incluindo o XGBoost. Isso pode resultar em um desempenho melhor para a LSTM em tarefas em que há uma grande quantidade de dados sequenciais.

Outra vantagem da LSTM é a sua capacidade de lidar com dados de entrada de diferentes tipos e tamanhos, como sequências de palavras, imagens e dados numéricos. Isso torna a LSTM uma escolha popular em aplicações em que os dados podem ter diferentes formatos, como reconhecimento de fala e tradução automática.

Para análise de sentimentos em reviews de usuários, a LSTM pode ser mais vantajosa do que o XGBoost. Isso ocorre porque as análises de sentimentos geralmente envolvem dados sequenciais, como sequências de palavras ou frases em uma revisão, e as LSTMs são projetadas especificamente para trabalhar com dados sequenciais e são capazes de aprender padrões em sequências de dados de maneira mais eficiente do que algoritmos de aprendizado de máquina.

Além disso, as LSTMs têm a capacidade de capturar dependências de longo prazo nos dados sequenciais, o que pode ser importante para a análise de sentimentos em reviews de usuários, já que as opiniões dos usuários muitas vezes são expressas em frases e parágrafos complexos e detalhados.

No entanto, é importante ressaltar que a escolha do algoritmo de \textit{machine learning} depende do conjunto de dados e do problema em questão. Algoritmos mais simples podem ser mais eficazes em alguns cenários, enquanto algoritmos mais complexos, como redes neurais, podem ser necessários para lidar com dados mais complexos e variáveis. Em última análise, a escolha do algoritmo deve ser baseada em uma análise cuidadosa dos dados e das necessidades específicas do problema em questão.
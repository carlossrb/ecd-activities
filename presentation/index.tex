\documentclass[aspectratio=169]{Beamer} % Retire o aspectratio para adaptar a projetores antigos
\usepackage[utf8]{inputenc}		% Codificação do documento (conversão automática dos acentos)
\usepackage{color}				% Controle das cores
\usepackage{graphicx}			% Inclusão de gráficos
\usepackage[brazil]{babel}		% Idioma do documento
\usepackage{float} 				% Fixa tabelas e figuras no local exato
\usepackage{subfig}				% para inserir mais figuras
\usepackage{amsmath}			% para alguns formatos de equações
\usepackage{amssymb}			% para mais símbolos
\usepackage{animate}			% para gifs
\usepackage{pdflscape}			% Pdf landscape
%------	%------	%------	%------	%------	%------	%------	
\title[Especialização em Ciência de Dados e Big Data - ECD]{Análise de sentimentos em avaliações de clientes do \textit{e-commerce} nacional e comparação de métodos tradicionais de \textit{machine learning} com Redes Neurais \textit{Long Short Term Memory} (LSTM)}
%------	%------	%------	%------	%------	%------	%------	
\author[Carlos Magno de Brito]{{\Large Carlos Magno Santos Ribeiro de Brito}}
%------	%------	%------	%------	%------	%------	%------	
\institute[UFBA]{{\large Universidade Federal da Bahia}\\Departamento de Engenharia Elétrica e de Computação}
%------	%------	%------	%------	%------	%------	%------	
\date[UFBA 2018]

\logo{\includegraphics[width=0.8 cm]{logo_ufba.eps}}
\newcommand{\nologo}{\setbeamertemplate{logo}{}} % command to set the logo to nothing
%------	
\usetheme{CambridgeUS}
\usecolortheme{beaver}
%------ %------	%------	%------	%------	%------	%------
\usefonttheme[onlymath]{serif}	
%------ %------	%------	%------	%------	%------	%------
\begin{document}
% Definindo a capa
\frame{\titlepage}
% Definindo o sumário dividido na quantidade de slides necessária
\frame{\tableofcontents[part=1]}
\frame{\tableofcontents[part=2]}
%------ %------	%------	%------	%------	%------	%------		
\part{1}
\section{Introdução}
\nologo
\frame{
    \frametitle{Introdução}
    \begin{itemize}
        \item Busca de um modelo simplificado que represente com precisão suficiente a estrutura; \vskip1cm
        \item Modelos mais robustos foram surgindo com o passar dos anos: mais graus de liberdade, maior discretização do modelo, etc.;\vskip1cm
        \item Maior processamento computacional aumentaram as possibilidades de utilização;\vskip1cm
        \item Método dos Elementos Finitos (MEF) -- \textit{Finite Element Method (FEM)}
    \end{itemize}
}
\frame{
    \frametitle{Introdução}
    \begin{itemize}
        \item Realiza-se a análise estática convencional e análise dinâmica:\vskip1cm
              \begin{itemize}
                  \item \textbf{Estática:} Forças internas dependem apenas da rigidez do material e não variam com o tempo
                        \begin{equation}
                            f=ku
                        \end{equation}
                  \item \textbf{Dinâmica:} Forças variantes com o tempo e dependem das forças inerciais e dissipativas
                        \begin{equation}
                            f(t)=ma+cv+ku
                        \end{equation}
              \end{itemize}
        \item Não linearidades geométrica e física;
    \end{itemize}
}
\subsection{Objetivos e Justificativa}
\frame{
    \frametitle{Objetivos e Justificativa}
    \begin{block}{Objetivo principal}
        Realizar estudos numéricos não lineares estáticos e dinâmicos
    \end{block}
    \begin{block}{Objetivo principal}
        Implementar e simular elementos de treliça a fim de ter uma biblioteca capaz de reproduzir diferentes tipos de treliça
    \end{block}
    \begin{alertblock}{Objetivo secundário}
        \begin{itemize}
            \item 	Avaliar o comportamento não linear estático e dinâmico das treliças;
            \item Avaliar diferentes tipos de carregamento;
            \item Influência do amortecimento proporcional de Rayleigh.
        \end{itemize}
    \end{alertblock}
}
\frame{
    \frametitle{Objetivos e Justificativa}
    Dentre as principais justificativas para o estudo não linear de treliças, tem-se:\vskip0.5cm
    \begin{itemize}
        \item Treliças com menos barras, mais esbeltas e mais eficientes;\vskip1cm
        \item Verificar diversas características de elementos mais complexo de forma menos custosa e vantajosa;\vskip1cm
        \item Compreender e propor um estudo dinâmico acoplado a esta formulação.
    \end{itemize}
}

\section{Conclusões}
\frame{
    \frametitle{Conclusões}
    \begin{itemize}
        \item Forma vetorial, se mostra como uma excelente alternativa aos métodos convencionais já consagrados;\vskip0.5cm
        \item Método de newton impõe limites que podem ser superados com controle de carga e deslocamento ao mesmo tempo;\vskip0.5cm
        \item Boa convergência em caso amortecido e não e intervalos diferenciados de tempo;\vskip0.5cm
        \item É interessante analisar a presença do amortecimento na formulação em comparação com um estudo de um modelo real simplificado;\vskip0.5cm
        \item Diferenças entre frequências atualizadas e não atualizadas são pequenas, mas podem causar o fenômeno de ressonância na estrutura caso desconsideradas no modelo.
    \end{itemize}
}

\end{document}
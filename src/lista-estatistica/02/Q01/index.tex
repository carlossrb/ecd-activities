
\noindent \textcolor{COLOR1}{Questão 01)} Por engano misturaram-se quatro pilhas novas com três pilhas usadas. Escolhendo ao acaso, e sem  reposição, duas dessas pilhas, determine a probabilidade uma ser nova e outra usada.
\\

Temos que inicialmente total de pilhas é $n = 7$. Quando não há reposição (eventos dependentes), a probabilidade da primeira tentativa ser nova é dada como $P(N) = \frac{4}{7}$, mas a probabilidade da segunda ser usada é $P(U|N) = \frac{3}{6} = 0.5$, pois o espaço amostral total diminuiu. A probabilidade de uma ser nova e outra usada pode ser definida como $P(N\cap U)=P(N)\times P(U|N) = \frac{4}{7}\times \frac{3}{6} \approx 28,6\%$.\\
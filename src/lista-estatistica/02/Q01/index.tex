
\noindent \textcolor{COLOR1}{Questão 01)} Por engano misturaram-se quatro pilhas novas com três pilhas usadas. Escolhendo ao acaso, e sem  reposição, duas dessas pilhas, determine a probabilidade uma ser nova e outra usada.
\\

Quando não há reposição (eventos dependentes), a probabilidade da primeira tentativa ser nova é dada como $P(N) = \frac{4}{7}$ e a probabilidade da segunda ser usada é $P(U|N) = \frac{3}{6}$, pois o espaço amostral total diminuiu. A probabilidade de uma ser nova e outra usada pode ser definida como $P(N\cap U)=P(N)\times P(U|N) = \frac{4}{7}\times \frac{3}{6} = \frac{2}{7}$.\\

Igualmente, considerando também o caso inverso, onde a primeira tentativa seria uma pilha usada e a segunda uma nova, temos  $P(U\cap N)=\frac{3}{7}\times \frac{4}{6} = \frac{2}{7}$.
\\

Logo, para o caso geral, soma-se ambos \textcolor{COLOR2}{$P(G) = \frac{2}{7} + \frac{2}{7} = \frac{4}{7}$}
\\


\noindent \textcolor{COLOR1}{Questão 04)} Um grupo de pesquisadores pretende estudar o tempo médio que um certo medicamento demora a fazer efeito. Com base numa amostra de $20$ pacientes obteve-se um tempo médio de $60$ minutos e uma variância de $100$ minutos.
\\

a) Qual a estimativa pontual do tempo médio que o medicamento demora a fazer efeito?
\\

A estimativa pontual já está dispoinível no enunciado. Ela equivale a \textcolor{COLOR2}{$\overbar{X}=60$ minutos}.
\\

b) Obtenha um intervalo com $98\%$ de confiança para o tempo médio que o medicamento demora a fazer efeito.
\\

Por se tratar de uma amostra pequea $n<30$, será utilizada a distribuição de t-student como modelo proposto. A quantidade de graus de liberdade é dada por $n-1=19$. O nível de confiança é definido como $1-\alpha = 0,98$.\\

Utilizando python, obtém-se então os valores de $t_0$ e $t_1$ onde que são os limites inferior e superior do intervalo de confiança e onde se espera encontrar o valor estimado do tempo médio com uma confiança de $98\%$.\\

\begin{lstlisting}
from scipy.stats import t

#Para uma distribuicao bicaudal simetrica

alpha = 0.02
n=20
df=n-1
    
a0 = alpha/2
a1 = 1 - a0
    
a = t(df).ppf((a0, a1))
    
print(a)
    
# [-2.53948319  2.53948319]
\end{lstlisting}

A partir desses valores de $t\approx2,539$, $\overbar{X}=60$ min, $S=\sqrt{100}$ min, calculamos o intervalo de confiança como:\\

\[
    \begin{split}
        &Ic(\mu; 1-\alpha) = \biggl[\overbar{X}+\textcolor{deepblue}{t_0\frac{S}{\sqrt{n}}};\overbar{X}+\textcolor{deepblue}{t_1\frac{S}{\sqrt{n}}}\biggr]\\
        &Ic(\mu; 0,98) = \biggl[60-2,539\times\frac{10}{\sqrt{20}};60+2,539\times\frac{10}{\sqrt{20}}\biggr]\\
        &\textcolor{COLOR2}{Ic(\mu; 0,98) = [54,322;65,678]}
    \end{split}
\]
\\

Onde a parcela em \textcolor{deepblue}{azul} representa o erro amostral associado.\\

c) Com base no intervalo de confiança obtido em (b), qual o erro amostral da estimativa pontual?\\

Conforme definido na questão anterior, o erro amostral pode ser calculado como:\\

\[
    Err = t_0\frac{S}{\sqrt{n}} = \textcolor{COLOR2}{5,6784\ min}
\]
\\

d) Estes pesquisadores decidiram recolher uma segunda amostra com $40$ pacientes, resultando num tempo médio de $50$ minutos e desvio padrão de $90$ minutos. Qual deverá ser o tamanho da amostra para que o erro cometido ao estimarmos o tempo médio que o medicamento demora a fazer efeito, não seja superior a $10$ minutos, com probabilidade $0.95$?\\

A partir da expressão do erro amostral, podemos deduzir uma equação para obter o tamanho da amostra. Além disso, é necessário calcular o $z_0$, dado a probabilidade de $0,95$ e erro máx de $Err=10$ min.\\
\\

\begin{equation}
    Err = z_0\frac{S}{\sqrt{n}} \to n\geqslant \biggl(\frac{z_0}{Err}\biggr)^2 S^2
\end{equation}
\\

Para o cálculo $z_0$, com auxílio do python:\\

\begin{lstlisting}
from scipy.stats import norm
    
alpha = 0.05
            
a0 = alpha/2
a1 = 1 - a0
    
a = norm.ppf((a0,a1))
            
print(a)
    
# [-1.95996398  1.95996398]
\end{lstlisting}

Substituindo os valores na Equação 1, obtemos:\\

\[
    n\geqslant \biggl(\frac{1,96}{10}\biggr)^2\times 90^2 \to n \geqslant \textcolor{COLOR2}{312\ \text{pacientes}}
\]
\\

\begin{center}
    \begin{tikzpicture}
        \draw (3.8,-0.3) node [anchor=west] [font=\footnotesize]  {$z_0\approx-1,96$};
        \draw (9,-0.3) node [anchor=west] [font=\footnotesize]  {$z_1\approx1,96$};
        \draw (5.5,3) node [anchor=west] [font=\footnotesize]  {$1-\alpha=0,95$};
        \draw [-{To[scale=1.5]}](3.5,0.5) -- (3,0.5) node [anchor=east]  [font=\footnotesize]  {$\sfrac{\alpha}{2}=0,025$};
        \begin{axis}[no markers,
                domain=0:10, samples=100,
                title={Curva normal parametrizada},
                axis lines*=center,
                axis y line=none,
                height=8cm, width=15cm,
                xtick=\empty, ytick=\empty,
                enlargelimits=false, clip=false, axis on top,
                line width=1.5pt,
            ]
            \addplot [
                domain=-5:5,
                color=deepblue,
                smooth,
                line width=1.0pt,
            ] {gauss(0,1)} \closedcycle;
            \addplot [fill=pagecolor!50!COLOR1, draw=none, domain=-5:-1.959963] {gauss(0,1)} \closedcycle;
            \addplot [fill=pagecolor!50!COLOR1, draw=none, domain=1.959963:5] {gauss(0,1)} \closedcycle;
        \end{axis}
    \end{tikzpicture}
\end{center}

\noindent \textcolor{COLOR1}{Questão 05)} Um grupo de pesquisadores pretende estudar o tempo médio que um certo medicamento demora a fazer efeito. Com base numa amostra de $20$ pacientes obteve-se um tempo médio de $60$ minutos e uma variância de $100$ minutos.
\\

a) Qual o tamanho da amostra necessário para que o erro cometido na estimação seja no máximo $5$ minutos, com probabilidade $0,98$.
\\

O procedimento é semelhante ao da questão anterior letra d, porém com alguns valores diferentes.\\

Da mesma forma, se calcula o valor de $z_0$ com auxílio do python.\\

\begin{lstlisting}
from scipy.stats import norm
        
alpha = 0.02
                
a0 = alpha/2
a1 = 1 - a0
        
a = norm.ppf((a0,a1))
                
print(a)
        
# [-2.32634787  2.32634787]
\end{lstlisting}

Com auxílio da Equação 1, podemos encontrar o valor de $n$:
\\

\[
    n\geqslant \biggl(\frac{2,326}{5}\biggr)^2 \times 100 \to n\geqslant \textcolor{COLOR2}{22\ \text{pacientes}}
\]
\\

b) Foi recolhida uma segunda amostra de $30$ pacientes (grupo B) e obteve-se um tempo médio de $50$ minutos e uma variância de $90$ minutos. Verifique se o tempo médio do grupo A é inferior ao do grupo B. Considere um nível de confiança de $95\%$.
\\

Deve-se testar as seguintes hipóteses:

\begin{enumerate}
    \item $H_0\to \mu_a - \mu_b\geqslant 0$: O tempo médio do grupo A é superior ou igual ao do grupo B.
    \item $H_a\to \mu_a - \mu_b< 0$: O tempo médio do grupo A é inferior ao do grupo B.
\end{enumerate}

Trata-se de um teste unilateral à esquerda em que podemos utilizar a distribuição de t-student com o total de graus de liberdade definido como $n_l=n_a+n_b-2$ e um nível de significância $\alpha=5\%$.\\

\begin{equation}
    T = \frac{(\overbar{X_a}-\overbar{X_b})-(\mu_a-\mu_b)}{
    \sqrt{\frac{(n_a-1)S^2_{a}+(n_b-1)S^2_{b}}{n_a+n_b-2}}
    \times
    \sqrt{\frac{1}{n_a}+\frac{1}{n_b}}
    }
    \sim t_{n_a+n_b-2}
\end{equation}
\\

Utilizando do python, calcula-se a região de rejeição do teste:
\\

\begin{lstlisting}
from scipy.stats import t

na = 20
nb = 30
alpha = 0.05
df = na + nb - 2
    
a = t(df).ppf(alpha)
    
print(a)
    
# -1.6772241953450402
\end{lstlisting}

Por se tratar de uma distribuição unilateral, temos que a região de rejeição é dada como $RR = (-\infty; -1,677)$.\\

Para o cálculo da estatística de testes, utilizamos a Equação 2 com auxílio do python:\\

\begin{lstlisting}
from math import sqrt

xa = 60
xb = 50
na = 20
nb = 30
vara = 100
varb = 90
sa = sqrt(vara)
sb = sqrt(varb)
    
den = na + nb - 2
dev = sqrt(((na - 1) * vara + (nb - 1) * varb) / den)
sroot = sqrt((1 / na) + (1 / nb))
t = (xa - xb) / (dev * sroot)
print(t)
    
# 3.573740145180839
\end{lstlisting}

Por fim, tem-se as condições das regras de decisão:

\begin{enumerate}
    \item Sendo $t\in RR$, $H_0$ é rejeitada.
    \item Sendo $t\notin RR$, $H_0$ \textbf{não} é rejeitada.
\end{enumerate}

Logo, como $t\approx 3,57$ e não está na região de rejeição, $RR = (-\infty; -1,677)$, \textcolor{COLOR2}{\textbf{não} devemos rejeitar $H_0$}. Por outro lado, \textcolor{COLOR2}{devemos rejeitar a hipótese alternativa $H_a$}.\\
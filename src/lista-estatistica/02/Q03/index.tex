
\noindent \textcolor{COLOR1}{Questão 03)} Para a população masculina de um determinado país, com idades entre $18$ e $74$ anos, a pressão sistólica tem distribuição aproximadamente normal com média $129$ mmHg e desvio padrão $19.8$ mmHg. Considere que os níveis da pressão são normais quando menores que $130$ mmHg.
\\

a) Qual a probabilidade de um homem dessa população possuir pressão sistólica normal?\\

Para isso, esse homem deve ter níveis de pressão abaixo de $130$ mmHg. Usando a distribuição normal como modelo, tem-se:
\\

\[
    X \sim N(\mu = 129; \sigma^2=19,8^2)
\]
\\

Onde $X$ é a pressão sistólica. Para ela sendo $P(X<130)$, tem-se:
\\

\[
    P(X<130)=P\biggl(\frac{X-129}{19,8}<\frac{130-129}{19,8}\biggr) = P\biggl(Z<\frac{1}{19,8}\biggr)\approx \textcolor{COLOR2}{0,5201}
\]
\\

O resultado foi obtido com python, com o seguinte código:\\

\begin{lstlisting}
from scipy.stats import norm


x = norm(129,19.8).cdf(130) # Sem padronizar
z = norm(0,1).cdf(1/19.8) # Padronizada em Z
    
print(x," ", z)
    
# 0.5201400375890497
\end{lstlisting}

b)  Qual a probabilidade de um homem dessa população possuir hipertensão moderada (pressão sistólica entre $160$ e $179$ mmHg)?\\

Para este caso, tem-se:
\\

\[
    P(160\leqslant X\leqslant 179)=P\biggl(\frac{160-129}{19,8}\leqslant\frac{X-129}{19,8}\leqslant\frac{179-129}{19,8}\biggr)
\]
\[
    P\biggl(\frac{31}{19,8}\leqslant Z\leqslant\frac{50}{19,8}\biggr) = P\biggl(0\leqslant Z \leqslant \frac{50}{19,8}\biggr)-P\biggl(0\leqslant Z \leqslant \frac{31}{19,8}\biggr) \approx \textcolor{COLOR2}{0,0529}
\]
\\

O resultado foi obtido com python, com o seguinte código:\\

\begin{lstlisting}
from scipy.stats import norm

# A partir de menos infinito
z1 = norm(0, 1).cdf(50 / 19.8)
z2 = norm(0, 1).cdf(31 / 19.8)
    
print(z1 - z2)
    
# 0.05293376095968105
\end{lstlisting}

c)  Selecionando ao acaso 1000 homens dessa população, quantos seriam diagnosticados com hipertensão moderada (pressão sistólica entre $160$ e $179$ mmHg)?\\

Tendo já obtido os valores de pressão sistólica para um conjunto ao qual pertece esses 1000 homens, basta apenas realizar a multiplicação:
\\

\[
    n=1000\times P(160\leqslant X\leqslant 179) \approx \textcolor{COLOR2}{53}
\]

d) Qual a probabilidade de um homem dessa população possuir pressão sistólica igual a $180$ mmHg?\\

Como o modelo de distribuição é contínuo $\int_{a}^{b} f(x)\,dx $, onde as probabilidades são dadas como áreas sob a curva gaussiana, um ponto infinitesimal $dx$ específico terá uma área aproximadamente igual a 0. Ou seja, \textcolor{COLOR2}{$P(X=180)=0$}. \\

e) Qual a pressão sistólica que deixa acima $80\%$ dos indivíduos?\\

Para $P(Z<z_0)=0,2$, onde $0,2$ corresponde a área acumulada na curva normal. Então, precisamos encontrar o valor de $z_0$, para em seguida achar o de $x_0$. Fazendo a transformação inversa da curva normal, tem-se:
\\

\[
    z_0=\frac{x_0-\mu}{\sigma}\to x_0 = z_0\sigma+\mu
\]
\\

O resultado pode ser obtido com o python e é igual a $112,34$ mmHg.
\\

\begin{lstlisting}
from scipy.stats import norm
mu = 129
sigma = 19.8
z0 = norm.ppf(0.2)
    
print(z0*sigma+mu)
    
# 112.33589957525629
\end{lstlisting}


\begin{center}
    \begin{tikzpicture}
        \draw (3,-0.3) node [anchor=west] [font=\footnotesize]  {$z_0\approx-0.842\to x_0\approx112,34$ mmHg};
        \draw [-{To[scale=1.5]}](4,1) -- (3,1) node [anchor=east]  [font=\footnotesize]  {$P(Z<z_0)=0,2$};
        \begin{axis}[no markers,
                domain=0:10, samples=100,
                title={Curva normal parametrizada},
                axis lines*=center,
                height=8cm, width=15cm,
                xtick=\empty, ytick=\empty,
                enlargelimits=false, clip=false, axis on top,
                line width=1.5pt,
            ]
            \addplot [
                domain=-5:5,
                color=deepblue,
                smooth,
                line width=1.0pt,
            ] {gauss(0,1)} \closedcycle;
            \addplot [fill=pagecolor!50!COLOR1, draw=none, domain=-5:-0.84] {gauss(0,1)} \closedcycle;
        \end{axis}
    \end{tikzpicture}
\end{center}



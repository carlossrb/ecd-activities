
\noindent \textcolor{COLOR1}{Questão 02)} Sabe-se que $5\%$ das pessoas que começam uma dieta têm disturbio alimentar. Ao selecionar, ao acaso,
$50$ pessoas em dieta, determine:
\\

a) A probabilidade de que pelo menos uma pessoa sofra de distúrbio alimentar;
\\

Para resolver este problema, podemos tomar a distribuição binomial como modelo a ser utilizado. A probabilidade de disturbio alimentar é dada como $p=\frac{5}{100}=\frac{1}{20}$ e $n=50$. Definindo $X$ como o número de pessoas com distúrbio alimentar, $X \thicksim  b(n=50;p=\frac{1}{20})$.
\\

\[
    P(X=x)=\binom{n}{x}p^x(1-p)^{n-x}
\]
\[
    P(X\geqslant 1)=\binom{50}{x}\biggl(\frac{1}{20}\biggr)^x\biggl(1-\frac{1}{20}\biggr)^{50-x}, \hspace{0.5cm} x=1,2,3,\dots ,50
\]
\\

Isso é o mesmo que $P(X\geqslant 1)=1-P(X=0)$. Para o valor de $P(X=0)$, temos:\\

\[
    P(X=0)=\binom{50}{0}\biggl(\frac{1}{20}\biggr)^0\biggl(1-\frac{1}{20}\biggr)^{50}\approx 0,0769
\]
\[
    P(X\geqslant 1)=1-0,0769 \approx 0,9231
\]

Ou seja, \textcolor{COLOR2}{$92,31\%$}.
\\

Obtém-se com Python os valores para a distribuição binomial com o seguinte código e plota-se o gráfico dessa distribuição:\\

\begin{lstlisting}
from scipy.stats import binom

n = 50
p = 0.05
x = range(0, n + 1)
    
a = binom.pmf(x, n, p)
for i, o in enumerate(a):
    print("{} {:.4f}".format(i,o*100))
\end{lstlisting}

\begin{center}
    \begin{tikzpicture}
        \begin{axis}[
                title={ Gráfico da f.p. $p(x)$ para $n = 50$ e $p = \sfrac{1}{20}$},
                xlabel={$x$},
                ylabel={$P(X)$},
                xmin=0, xmax=50,
                ymin=0, ymax=30,
                xtick={0,10,20,30,40,50},
                ytick={0,5,10,15,20,25,30},
                ymajorgrids=true,
                grid style=dashed,
                grid=both,
                grid=both,
                width=15cm,
                height=8cm,
            ]
            \addplot[
                color=COLOR2,
                smooth,
                line width=1.5pt,
            ]
            coordinates {
                    (0,7.6945)
                    (1,20.2487)
                    (2,26.1101)
                    (3,21.9875)
                    (4,13.5975)
                    (5,6.5841)
                    (6,2.5990)
                    (7,0.8598)
                    (8,0.2432)
                    (9,0.0597)
                    (10,0.0129)
                    (11,0.0025)
                    (12,0.0004)
                    (13,0.0001)
                    (14,0.0000)
                    (15,0.0000)
                    (16,0.0000)
                    (17,0.0000)
                    (18,0.0000)
                    (19,0.0000)
                    (20,0.0000)
                    (21,0.0000)
                    (22,0.0000)
                    (23,0.0000)
                    (24,0.0000)
                    (25,0.0000)
                    (26,0.0000)
                    (27,0.0000)
                    (28,0.0000)
                    (29,0.0000)
                    (30,0.0000)
                    (31,0.0000)
                    (32,0.0000)
                    (33,0.0000)
                    (34,0.0000)
                    (35,0.0000)
                    (36,0.0000)
                    (37,0.0000)
                    (38,0.0000)
                    (39,0.0000)
                    (40,0.0000)
                    (41,0.0000)
                    (42,0.0000)
                    (43,0.0000)
                    (44,0.0000)
                    (45,0.0000)
                    (46,0.0000)
                    (47,0.0000)
                    (48,0.0000)
                    (49,0.0000)
                    (50,0.0000)
                };

        \end{axis}
    \end{tikzpicture}
\end{center}


b) O número médio e o desvio padrão das pessoas com distúrbio alimentar.\\

A média e o desvio padrão são calculados como:\\

\[
    E(X) = np = 50\times\frac{1}{20} = \textcolor{COLOR2}{2,5}
\]
\[
    \sigma = \sqrt{Var(X)} = \sqrt{npq} = \sqrt{50 \times \frac{1}{20} \times \biggl(1-\frac{1}{20}\biggr)} \approx \textcolor{COLOR2}{1,54}
\]

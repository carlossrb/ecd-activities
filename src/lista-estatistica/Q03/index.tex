
\noindent \textcolor{COLOR1}{Questão 03)} Um inquérito por amostragem sobre o número de filhos em famílias de baixo escalão de
rendimento permitiu obter os seguintes números:
\\

\[
    \{ 3, 5, 4, 2, 3, 3, 5, 8, 7, 5, 4, 5, 6, 3, 4, 5, 4, 6, 7, 6\}
\]
\\

a) Determine a média, a moda e a variância a partir dos dados não classificados.
\\

\[N=20\]

\[
    media = \frac{3+5+4+2+3+3+5+8+7+5+4+5+6+3+4+5+4+6+7+6}{20} =
\]

\[
    =\frac{95}{20} = 4,75
\]
\\

para a moda, pode-se plotar um histograma e verificar o maior número de ocorrências:

\begin{center}
    \begin{tikzpicture}
        \begin{axis}
            [
                % boxplot/draw direction=y,
                ybar interval,
                height=8cm,
                width=13cm,
                xlabel=Número de filhos,
                ylabel=Frequência abs,
            ]
            \addplot+[
                color=deepblue,
                mark options={deepblue},
                fill = pagecolor!50!COLOR1,
                hist={bins=6}
            ] table [row sep=\\,y index=0] {3\\ 5\\ 4\\ 2\\ 3\\ 3\\ 5\\ 8\\ 7\\ 5\\ 4\\ 5\\ 6\\ 3\\ 4\\ 5\\ 4\\ 6\\ 7\\ 6 \\};
        \end{axis}
    \end{tikzpicture}
\end{center}

Observa-se que o valor mais frequente é o 5. Logo, a moda é 5.
\\

Para a variância, tem-se:
\\

\[
    s^2 = \frac{\sum_{n = 1}^{20} \left(x_i-\bar{x}\right)^2}{n-1} \approx 2,51
\]
\\

d) Classifique (i.e. agrupe) os dados e verifique os valores calculados em (a). Que conclusões se podem tirar?
\\

Ordenando...

\[
    \{ 2,3,3,3,3,4,4,4,4,5,5,5,5,5,6,6,6,7,7,8\}
\]
\\

Conforme visto no histograma da questão anterior, as famílias têm 5 filhos com maior frequência e poucas famílias de baixa renda têm poucos filhos (2). Para uma análise mais aprofundada desses dados, as faixas de rendimento também poderiam ser relacionadas.
\\


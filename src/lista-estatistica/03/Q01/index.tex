
\noindent \textcolor{COLOR1}{Questão 01)}  Pretende-se, se possível, modelar através de uma reta de regressão linear simples a quantidade de vidro $Y$ produzido num ponto de reciclagem (Kg), usando como variável independente x o número de dias sem despejar o mesmo. Para tal, registaram-se os seguintes dados.
\\

\begin{table}[ht]
    \centering
    \begin{tabular}{c|c|c|c|c|c|c|c|c}
        \rowcolor{pagecolor!50!COLOR1}
        \hline
        $x_i$ & 2   & 3   & 4 & 5   & 10  & 15  & 20   & 25   \\ \hline\hline
        $Y_i$ & 100 & 150 & - & 320 & 650 & 810 & 1040 & 1480 \\\hline
    \end{tabular}
\end{table}


O valor de $Y$ para $x = 4$ foi perdido, mas antes foram obtidos os seguintes resultados com base nos dados originais:
\\

\[
    \sum_{i=1}^{8}x_i=84\ \ \sum_{i=1}^{8}Y_i=4800\ \ \sum_{i=1}^{8}x_i^2=1404\ \ \sum_{i=1}^{8}Y_i^2=4548000\ \ \sum_{i=1}^{8}x_iY_i=79700
\]
\\
a) Escreva a reta de regressão estimada através do método dos mínimos quadrados.\\

Precisaremos encontrar os valores de $\overbar{x}\ \overbar{Y}\ \widehat{\beta_1} \ \text{e}\ \widehat{\beta_0}$ para assim formar a reta de regressão $Y=\widehat{\beta_1}x + \widehat{\beta_0}$:
\\

\[
    \begin{split}
        & \overbar{x} = \frac{1}{8} \sum_{i=1}^{8}x_i = \frac{84}{8} = 10,5     \\
        & \overbar{Y} = \frac{1}{8} \sum_{i=1}^{8}Y_i = \frac{4800}{8} = 600\\
        & \widehat{\beta_1} = \frac{\sum_{i=1}^{8}x_iY_i - n\overbar{x}\overbar{Y}}{\sum_{i=1}^{8}x^2_i - n\overbar{x}^2} = \frac{79700 - 8\times10,5\times600}{1404 - 8\times10,5^2} \approx 56,1303\\
        & \widehat{\beta_0} = \overbar{Y} - \widehat{\beta_1} \overbar{x} = 600 - 56,1303 \approx 10,6322\\
        & \textcolor{COLOR2}{Y = 56,1303x + 10,6322}
    \end{split}
\]
\\

\noindent b) Acha que conseguiu um bom ajuste? Use o coeficiente de determinação.\\

Tem-se que o coeficiente de determinação é $R^2$ é dado como:\\

\[
    R^2 = \frac{\bigl(\widehat{\beta_1}^2\bigr)\sum_{i=1}^{8}x_i^2-n\overbar{x}^2}{\sum_{i=1}^{8}Y_i^2-n\overbar{Y}^2} = \frac{56,1303^2\times(1404-8\times 10,5^2)}{4548000-8\times600^2} \textcolor{COLOR2}{\approx 0,986}
\]
\\

Dado o valor de $R^2$ obtido muito próximo de 1, pode-se afirmar que o ajuste foi bem sucedido.\\

\noindent c) Qual o valor da quantidade de vidro produzida no ponto de reciclagem que prevê ocorrer em 28 dias sem o despejar?\\

Não é possível extrapolar valores fora do intervalo usado para o ajuste. Assim sendo, para 28 dias, \textcolor{COLOR2}{não se pode prever o valor de $Y$ produzido.}\\


\noindent d) Teste se o declive da reta de regressão obtida em (a) é zero, usando um nível de significância de $10\%$. Como interpreta a não rejeição dessa hipótese?\\

Temos as seguintes hipóteses:
\begin{itemize}
    \item $H_0: \beta_1 = 0$
    \item $H_a: \beta_1 \neq 0$
\end{itemize}

Cálculo da variância do erro:

\[
    \begin{split}
        &\widehat{\sigma}^2 = \frac{1}{n-2}\biggl\{ \biggl(\sum_{i=1}^{8}Y_i^2 - nY^2\biggr)- \bigl(\widehat{\beta_1}\bigr)^2\biggl(\sum_{i=1}^{8}x_i^2 - n\overbar{x}^2\biggr)\biggr\}\\
        &\widehat{\sigma}^2 = \frac{1}{6}\{(4548000-8\times600^2) - 56,1303^2\times(1404-8\times 10,5^2)\} \approx 3896,8797
    \end{split}
\]

Cálculo da estatística de teste $T$:

\[
    \begin{split}
        &T = \frac{\widehat{\beta_1} - \beta_1 }{\sqrt{\frac{\widehat{\sigma}^2}{\sum_{i=1}^{8}x_i^2 - n\overbar{x}^2}}} \sim t_{n-2}\\ \\
        &T = \frac{56,1303 - 0}{\sqrt{\frac{3896,8797}{1404-8\times10,5^2}}} \sim t_{6} \approx 20,5356
    \end{split}
\]

Para a região crítica com $gl=6$ e $10\%$ de significância, temos o valor de $t_0 = 1,943$ (tabelado ou no python). Assim sendo, a região de rejeição será:


\[
    RR = ]-\infty; -1,943[U]1,943;+\infty[
\]
\\

Então, como o valor de $T$ obtido é $20,5356$, está dentro da região de rejeição, \textcolor{COLOR2}{devemos rejeitar a hipótese nula $H_0$}. Isso quer dizer que o valor de $\beta_1$ é considerável na inclinação e não pode ser desconsiderado.\\

\noindent e) Qual o erro de previsão quando o ponto de reciclagem não é despejado durante 10 dias?\\

Para o caso, basta achar o valor previsto pelo ajuste e subtrair pelo valor real medido:

\begin{itemize}
    \item Estimado $\to Y^e_{10} = 56,1303x_{10} + 10,6322 = 56,1303\times10 + 10,6322 = 571,9352$\\
    \item Tabelado $\to Y^t_{10} = 650$\\
    \item Diferença $\to \|  \Delta Y_{10}\|  = 571,9352 - 650 = \textcolor{COLOR2}{78,0648}$\\
\end{itemize}



\noindent \textcolor{COLOR1}{Questão 01)}  Pretende-se, se possível, modelar através de uma reta de regressão linear simples a quantidade de vidro $Y$ produzido num ponto de reciclagem (Kg), usando como variável independente x o número de dias sem despejar o mesmo. Para tal, registaram-se os seguintes dados.
\\

\begin{table}[ht]
    \centering
    \begin{tabular}{c|c|c|c|c|c|c|c|c}
        \rowcolor{pagecolor!50!COLOR1}
        \hline
        $x_i$ & 2   & 3   & 4 & 5   & 10  & 15  & 20   & 25   \\ \hline\hline
        $Y_i$ & 100 & 150 & - & 320 & 650 & 810 & 1040 & 1480 \\\hline
    \end{tabular}
\end{table}


O valor de $Y$ para $x = 4$ foi perdido, mas antes foram obtidos os seguintes resultados com base nos dados originais:
\\

\[
    \sum_{i=1}^{8}x_i=84\ \ \sum_{i=1}^{8}Y_i=4800\ \ \sum_{i=1}^{8}x_i^2=1404\ \ \sum_{i=1}^{8}Y_i^2=4548000\ \ \sum_{i=1}^{8}x_iY_i=79700
\]
\\
a) Escreva a reta de regressão estimada através do método dos mínimos quadrados.\\
as
\\
b) Acha que conseguiu um bom ajuste? Use o coeficiente de determinação.\\
c) Qual o valor da quantidade de vidro produzida no ponto de reciclagem que prevê ocorrer em 28 dias sem o despejar?\\
d) Teste se o declive da reta de regressão obtida em (a) é zero, usando um nível de significância de $10\%$. Como interpreta a não rejeição dessa hipótese?\\
e) Qual o erro de previsão quando o ponto de reciclagem não é despejado durante 10 dias?\\
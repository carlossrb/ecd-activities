
\noindent \textcolor{COLOR1}{Questão 02)} Considere o conjunto de dados ``Wage'' do pacote ``ISLR2'' do software R. Considere a variável ``health\_ins'' como variável resposta e as variáveis ``age'', ``maritl'', ``race'', ``education'', ``jobclass'', ``health'' e ``logwage'' como variáveis explicativas. Ajuste uma regressão logística, escreva o modelo final e interprete os coeficientes obtidos.
\\

Rodando o seguinte script, temos:
\\

\begin{lstlisting}
install.packages("ISLR2")
library(ISLR2)

summary(glm(health_ins ~ age + maritl + race + education + jobclass
 + health + logwage, data = Wage, family = binomial))
\end{lstlisting}

\begin{lstlisting}
    Call:
glm(formula = health_ins ~ age + maritl + race + education + 
    jobclass + health + logwage, family = binomial, data = Wage)

Deviance Residuals: 
    Min       1Q   Median       3Q      Max  
-2.0259  -0.8050  -0.5755   0.9413   2.8857  

Coefficients:
                         beta Std. Error z value Pr(>|z|)    
(Intercept)              12.345276   0.768926  16.055  < 2e-16 ***
age                      -0.016221   0.004311  -3.763 0.000168 ***
maritl2. Married          0.277127   0.119370   2.322 0.020255 *  
maritl3. Widowed         -0.171424   0.575412  -0.298 0.765767    
maritl4. Divorced        -0.129080   0.204525  -0.631 0.527960    
maritl5. Separated        0.277519   0.320369   0.866 0.386354    
race2. Black              0.059940   0.144589   0.415 0.678468    
race3. Asian              0.316402   0.181673   1.742 0.081579 .  
race4. Other              0.111366   0.366256   0.304 0.761078    
education2. HS Grad      -0.406558   0.150140  -2.708 0.006772 ** 
education3. Some College -0.517576   0.165375  -3.130 0.001750 ** 
education4. College Grad -0.463202   0.172532  -2.685 0.007259 ** 
education5. Advanced Deg -0.308215   0.205246  -1.502 0.133179    
jobclass2. Information   -0.349047   0.091890  -3.799 0.000146 ***
health2. >=Very Good     -0.144779   0.096927  -1.494 0.135256    
logwage                  -2.618243   0.175277 -14.938  < 2e-16 ***
---
Signif. codes:  0 '***' 0.001 '**' 0.01 '*' 0.05 '.' 0.1 ' ' 1

(Dispersion parameter for binomial family taken to be 1)

    Null deviance: 3693.5  on 2999  degrees of freedom
Residual deviance: 3182.7  on 2984  degrees of freedom
AIC: 3214.7

Number of Fisher Scoring iterations: 4
\end{lstlisting}

tem-se então a equação de regressão logística que é:
\\
\begin{equation}
    P(x) = \frac{e^{\beta_0 + \beta_1 x_1 +\cdots + \beta_n x_n}}{1 + e^{\beta_0 + \beta_1 x_1 +\cdots + \beta_n x_n}}
\end{equation}

As chances da pessoa possuir um plano de saúde são aumentadas quando temos variáveis ($\beta$) positivas. O caso contrário também é válido, ou seja, as negativas diminuem a chance do indivíduo possuir plano de saúde.

\noindent \textcolor{COLOR1}{Questão 04)} Considere a seguinte tabela:
\\

\begin{table}[ht]
    \centering
    \begin{tabular}{l|l|l|l}
        \hline
        \rowcolor{pagecolor!50!COLOR1}
        \multicolumn{1}{c|}{\cellcolor{pagecolor!50!COLOR1}}                                    & \multicolumn{2}{l|}{\cellcolor{pagecolor!50!COLOR1}Doença no pulmão} & \cellcolor{pagecolor!50!COLOR1}                                                          \\ \cline{2-3}
        \rowcolor{pagecolor!50!COLOR1}
        \multicolumn{1}{c|}{\multirow{-2}{*}{\cellcolor{pagecolor!50!COLOR1}{Hábito de fumar}}} & Presente                                                             & Ausente                         & \multirow{-2}{*}{\cellcolor{pagecolor!50!COLOR1}Total} \\ \hline \hline
        Sim                                                                                     & 25                                                                   & 15                              & 40                                                     \\ \hline
        Não                                                                                     & 10                                                                   & 40                              & 50                                                     \\ \hline
        Total                                                                                   & 35                                                                   & 55                              & 90                                                     \\ \hline
    \end{tabular}
\end{table}


a) Verifique se existe associação entre o hábito de fumar e a presença ou ausência de doença do pulmão.
\\

Tem-se que em pessoas que fumam, a percentagem de doenças de pulmão igual a $\frac{25}{40} = 62,5\%$. A ausência equivale a $\frac{15}{40} = 37,5\%$.
\\

Nos não fumantes, as doenças de pulmão são equivalentes a $\frac{10}{50} = 20 \%$. A ausência equivale a $\frac{40}{50} = 80\%$.
\\

Nessa amostra, especificamente, foi possível constatar uma relação entre o hábito de fumar e a presença de doença do pulmão, já que entre os fumantes $62,5\%$ deles têm doença no pulmão, enquanto nos não fumantes há uma relação de $20\%$ de pessoas que têm doença do pulmão. Além disso, os não fumantes apresentam uma quantidade significativa de $80\%$ de pessoas sem doença pulmonar. Dadas essas circunstâncias, é razoável supor que o cigarro é o causador de doenças de pulmão nessa amostra populacional.\\

b) Qual a percentagem de fumantes?
\\

\[
    fumantes = \frac{40}{90} \approx 44,4 \%
\]
\\

c) Quantas pessoas são fumantes e não têm doença do pulmão?
\\

Total de 15 pessoas o que corresponde a $\frac{15}{90} \approx 16,67\% $ da população total.
\\



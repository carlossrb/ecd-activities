
\noindent \textcolor{COLOR1}{Questão 02)}  Uma análise às características de 20 alimentos, utilizando o software IBM SPSS 20, resultou nos
seguintes resultados:
\\


\begin{center}
    \begin{tikzpicture}
        \begin{axis}
            [
                % boxplot/draw direction=y,
                ytick={1,2,3,4,5},
                yticklabels={Energia (Kcal), Proteínas (g), Lipídios(g), Cálcio (mg), Ferro (mg)},
                width=0.6\textwidth,
                xmin=-50,
                xmax=1000,
            ]
            \addplot+[
                color=deepblue,
                mark=triangle*,
                mark options={deepblue},
                fill = pagecolor!50!COLOR1,
                boxplot prepared={
                        median=183,
                        upper quartile=361.25,
                        lower quartile=36.75,
                        upper whisker=770,
                        lower whisker=10
                    },
            ] table [row sep=\\,y index=0] {900\\};
            \addplot+[
                color=deepblue,
                mark=triangle*,
                mark options={deepblue},
                fill = pagecolor!50!COLOR1,
                boxplot prepared={
                        median=3.1,
                        upper quartile=18.750,
                        lower quartile=1.125,
                        upper whisker=22,
                        lower whisker=1
                    },
            ] coordinates {};
            \addplot+[
                color=deepblue,
                mark=asterisk,
                mark options={deepblue},
                fill = pagecolor!50!COLOR1,
                boxplot prepared={
                        median=0.950,
                        upper quartile=13.375,
                        lower quartile=0.275,
                        upper whisker=21,
                        lower whisker=0
                    },
            ] table [row sep=\\,y index=0] {25\\ 80\\ 100\\};
            \addplot+[
                color=deepblue,
                mark=star,
                mark options={deepblue},
                fill = pagecolor!50!COLOR1,
                boxplot prepared={
                        median=28,
                        upper quartile=125.75,
                        lower quartile=13.5,
                        upper whisker=250,
                        lower whisker=0
                    },
            ] table [row sep=\\,y index=0] {800\\ 805\\};
            \addplot+[
                color=deepblue,
                mark=triangle*,
                mark options={deepblue},
                fill = pagecolor!50!COLOR1,
                boxplot prepared={
                        median=1,
                        upper quartile=1.75,
                        lower quartile=0.275,
                        upper whisker=1.8,
                        lower whisker=0.1
                    },
            ] table [row sep=\\,y index=0] {2\\};
        \end{axis}
    \end{tikzpicture}
\end{center}

\begin{longtable}{l|c|c|c|c|c}
    \hline
    \rowcolor{pagecolor!50!COLOR1}
                    & Energia (Kcal) & Proteínas (g) & Lipídios (g) & Cálcio (mg) & Ferro (mg) \\ \hline\hline
    N - Valid       & 20             & 20            & 20           & 20          & 20         \\ \hline
    N - Missing     & 0              & 0             & 0            & 0           & 0          \\ \hline
    Mean            & 240,80         & 8,550         & 13,775       & 135,505     & 1,427      \\ \hline
    Median          & 183,00         & 3,100         & 0,950        & 28,00       & 1,000      \\ \hline
    Std. Deviation  & 246,687        & 9,2488        & 28,3894      & 235,9819    & 1,5640     \\ \hline
    Percentile (25) & 36,75          & 1,125         & 0,275        & 13,500      & 0,275      \\ \hline
    Percentile (50) & 183,00         & 3,100         & 0,950        & 28,00       & 1,000      \\ \hline
    Percentile (75) & 361,25         & 18,750        & 13,375       & 125,750     & 1,750      \\ \hline
\end{longtable}


a) Cite e classifique as variáveis em estudo.
\\

Em termos gerais, tem-se alimentos com valor energético bastante diverso. Isso é visto no \textit{boxplot}, onde temos casos variando de algo próximo de zero (limite inferior) a oitocentos (limite superior). Há um \textit{outlier} (azeite) que tem valor de energia próximo de mil. Ao menos metade dos alimentos apresentam valores energéticos no intervalo de 36 a 360 Kcal, que são os quartis 1 e 3. Além disso, tem-se o Cálcio que também apresenta valores com grande variação e com dois alimentos com elevado potencial (\textit{outliers}). Em relação Ferro, Proteínas e Lipídios, há pouca variação de valores. Isso fica evidente pelo \textit{boxplot} dessas variáveis. Há alimentos atípicos com valores de lipídios fora do limite superior, ainda assim.
\\

b) Qual das variáveis apresenta uma maior dispersão? Qual a medida utilizada para responder a tal pergunta?
\\

Tem-se bem claramente, em primeiro e segunto lugares, que a Energia e o Cálcio são as duas variáveis que apresentam maior dispersão. A medida utilizada para responder a essa pergunta é o Desvio Padrão, que em ambos os casos ultrapassa o valor de 200.
\\

c) É correto dizer que a quantidade média de cálcio é superior à quantidade média de proteínas? Porquê?
\\

Não. Temos médias representadas com unidades de medidas diferentes. Enquanto as proteínas são representadas em gramas, a quantidade de cálcio é representada em miligramas. Para tanto, se o valor de cálcio for convertido em gramas é possível ver que ele é bem menor que a quantidade de proteínas.
\\

d) A partir da tabela e figura acima, escreva um breve parágrafo com a interpretação e comparação dos resultados.
\\

O estudo apresenta resultados com unidades de medidas diferentes (g e mg) e isso pode confundir a priori caso se queira comparar valores de diferentes variáveis entre si. Atentando-se a esse detalhe, observa-se que as proteínas e lipídios são os dois que apresentam maiores valores numéricos. O ferro apresenta valores bastante baixos em todos os alimentos. E o cálcio, por fim, tem quantidades relativamente menores (dados em miligramas) e bem variável nos alimentos, indo de 0 a quase 1,000 miligramas. Em termos energéticos, vemos também uma grande variabilidade de valores bem evidentes no \textit{boxplot}.
\\
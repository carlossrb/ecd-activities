
\noindent \textcolor{COLOR1}{Questão 02)}  Uma análise às características de 20 alimentos, utilizando o software IBM SPSS 20, resultou nos
seguintes resultados:
\\

\tikzset{every picture/.style={line width=0.75pt}} %set default line width to 0.75pt        
\begin{center}
    \begin{tikzpicture}[x=0.75pt,y=0.75pt,yscale=1.5,xscale=1.5]

        %Shape: Axis 2D 
        \draw[->] [color={deepblue}  ,draw opacity=1 ][line width=1.5] (0,0) -- (100,0);
        \draw[->] [color={deepblue}  ,draw opacity=1 ][line width=1.5] (8,-8) -- (8,300)
        node[left] {$pop$};

        %Shape: Rectangle
        \draw   (-30,-30) -- (250,-30) -- (250,330) -- (-30,330) -- cycle ;

        %Shape: Rectangle
        \draw  [color={COLOR2}  ,draw opacity=1 ][fill={rgb, 255:red, 179; green, 28; blue, 28 }  ,fill opacity=0.07 ] (42,151.25) .. controls (40.9,151.25) and (40,150.35) .. (40,149.25) -- (40,68) .. controls (40,66.9) and (40.9,66) .. (42,66) -- (68,66) .. controls (69.1,66) and (70,66.9) .. (70,68) -- (70,149.25) .. controls (70,150.35) and (69.1,151.25) .. (68,151.25) -- cycle ;
        %Line 
        \draw [color={deepblue}  ,draw opacity=1 ]   (70,117) -- (40,117) ;

        %Straight Lines 
        \draw [color={COLOR2}  ,draw opacity=1 ] [dash pattern={on 4.5pt off 4.5pt}]  (55,275) -- (55,151.25) ;
        \draw [shift={(55,275)}, rotate = 450] [color={COLOR2}  ,draw opacity=1 ][line width=0.75]    (0,5.59) -- (0,-5.59)   ;
        %Straight Lines 
        \draw [color={COLOR2}  ,draw opacity=1 ] [dash pattern={on 4.5pt off 4.5pt}]  (55,66) -- (55,0) ;
        \draw [shift={(55,0)}, rotate = 450] [color={COLOR2}  ,draw opacity=1 ][line width=0.75]    (0,5.59) -- (0,-5.59)   ;


        % Text Node
        \draw [-{To[scale=1.5]}](75,117) -- (110,117) node [anchor=west]  [font=\footnotesize]  {$mediana\ =\ q_{2}$};
        % Text Node
        \draw [decorate,decoration={brace,amplitude=10pt,mirror,raise=4pt},yshift=0pt]
        (70,151.25) -- (70,275) node [midway,xshift=2.5cm] {\footnotesize
            $limite\ superior\ =\ L_{s}$};
        % Text Node
        \draw (8,5) node [anchor=west]  [font=\footnotesize]  {$0$};
        % Text Node
        \draw [-{To[scale=1.5]}](75,151.25) -- (110,151.25) node [anchor=west]  [font=\footnotesize]  {$terceiro\ quartil\ =\ q_{1}$};
        % Text Node
        \draw [-{To[scale=1.5]}](75,66) -- (110,66) node [anchor=west]  [font=\footnotesize]  {$primeiro\ quartil\ =\ q_{1}$};
        % Text Node
        \draw [decorate,decoration={brace,amplitude=10pt,mirror,raise=4pt},yshift=0pt]
        (70,0.65) -- (70,66) node [midway,xshift=2.5cm] {\footnotesize
            $limite\ inferior\ =\ L_{i}$};
        % Text Node
        \draw (8,275) node [anchor=west]  [font=\footnotesize]  {$100$};
        % Text Node
        \draw (8,151.25) node [anchor=west] [font=\footnotesize]  {$55$};
        % Text Node
        \draw (8,117) node [anchor=west] [font=\footnotesize]  {$42,5$};
        % Text Node
        \draw (8,66) node [anchor=west] [font=\footnotesize]  {$24$};
        % Text Node
        \draw (100,275) [color={COLOR2}  ,draw opacity=1 ] node [anchor=north west][inner sep=0.75pt]  [font=\small]  {Boxplot - Variação de idade};
        % ticks
        \foreach \y in {15,30,...,105} {%
                \draw [color={deepblue}] ($(6,\y*2.75) $) -- ($(10,\y*2.75)$)
                node [left] [font=\tiny, color={deepblue}, xshift=-1.5pt] {$\y$};
            }
    \end{tikzpicture}

\end{center}

\begin{longtable}{l|c|c|c|c|c}
    \hline
    \rowcolor{pagecolor!50!COLOR1}
                    & Energia (Kcal) & Proteínas (g) & Lipídios (g) & Cálcio (mg) & Ferro (mg) \\ \hline\hline
    N - Valid       & 20             & 20            & 20           & 20          & 20         \\ \hline
    N - Missing     & 0              & 0             & 0            & 0           & 0          \\ \hline
    Mean            & 240,80         & 8,550         & 13,775       & 135,505     & 1,427      \\ \hline
    Median          & 183,00         & 3,100         & 0,950        & 28,00       & 1,000      \\ \hline
    Std. Deviation  & 246,687        & 9,2488        & 28,3894      & 235,9819    & 1,5640     \\ \hline
    Percentile (25) & 36,75          & 1,125         & 0,275        & 13,500      & 0,275      \\ \hline
    Percentile (50) & 183,00         & 3,100         & 0,950        & 28,00       & 1,000      \\ \hline
    Percentile (75) & 361,25         & 18,750        & 13,375       & 125,750     & 1,750      \\ \hline
\end{longtable}


a) Cite e classifique as variáveis em estudo.
\\

Em termos gerais, tem-se alimentos com valor energético bastante diverso. Isso é visto no \textit{boxplot}, onde temos casos variando de algo próximo de zero (limite inferior) a oitocentos (limite superior). Há um \textit{outlier} (azeite) que tem valor de energia próximo de mil. Ao menos metade dos alimentos apresentam valores energéticos no intervalo de 36 a 360 Kcal, que são os quartis 1 e 3. Além disso, tem-se o Cálcio que também apresenta valores com grande variação e com dois alimentos com elevado potencial (\textit{outliers}). Em relação Ferro, Proteínas e Lipídios, há pouca variação de valores. Isso fica evidente pelo \textit{boxplot} dessas variáveis. Há alimentos atípicos com valores de lipídios fora do limite superior, ainda assim.
\\

b) Qual das variáveis apresenta uma maior dispersão? Qual a medida utilizada para responder a tal pergunta?
\\

Tem-se bem claramente, em primeiro e segunto lugares, que a Energia e o Cálcio são as duas variáveis que apresentam maior dispersão. A medida utilizada para responder a essa pergunta é o Desvio Padrão, que em ambos os casos ultrapassa o valor de 200.
\\

c) É correto dizer que a quantidade média de cálcio é superior à quantidade média de proteínas? Porquê?
\\

Não. Temos médias representadas com unidades de medidas diferentes. Enquanto as proteínas são representadas em gramas, a quantidade de cálcio é representada em miligramas. Para tanto, se o valor de cálcio for convertido em gramas é possível ver que ele é bem menor que a quantidade de proteínas.
\\

d) A partir da tabela e figura acima, escreva um breve parágrafo com a interpretação e comparação dos resultados.
\\

O estudo apresenta resultados com unidades de medidas diferentes (g e mg) e isso pode confundir a priori caso se queira comparar valores de diferentes variáveis entre si. Atentando-se a esse detalhe, observa-se que as proteínas e lipídios são os dois que apresentam maiores valores numéricos. O ferro apresenta valores bastante baixos em todos os alimentos. E o cálcio, por fim, tem quantidades relativamente menores (dados em miligramas) e bem variável nos alimentos, indo de 0 a quase 1,000 miligramas. Em termos energéticos, vemos também uma grande variabilidade de valores bem evidentes no \textit{boxplot}.
\\
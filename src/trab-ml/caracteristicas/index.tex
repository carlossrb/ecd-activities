
\noindent \textcolor{COLOR2}{PRINCIPAIS CARACTERÍSTICAS}
\\

Para o tratamento desses dados foi utilizada a linguagem python, que é uma linguagem de programação de alto nível, com uma sintaxe simples e flexível.\\

Além disso, conforme dito na introdução, tem-se um conjunto de dados onde estão disponíveis dados de 306 pacientes submetidos à cirurgia de câncer de mama. Esse \textit{dataset} contém as seguintes informações:
\begin{itemize}
    \item \textit{Idade} - Idade do paciente no momento da cirurgia variando de 30 a 83 anos;
    \item \textit{Ano} - Ano em que o paciente foi submetido à cirurgia variando entre 1958 e 1969;
    \item \textit{Número de nódulos} - Número de nódulos encontrados na mama variando entre 0 e 52;
    \item \textit{Status de sobrevivência} - Com status igual a 1, o paciente sobreviveu 5 anos ou mais; com status igual a 2, o paciente morreu dentro de 5 anos;
\end{itemize}

Com essas quatro variáveis se analisou o dataset e foi possível identificar qual melhor modelo melhor se adequa na classificação desses resultados.\\
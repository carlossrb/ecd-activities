
\noindent \textcolor{COLOR2}{INTRODUÇÃO}
\\


O câncer é um problema de saúde que se caracteriza por uma mortalidade anual bastante elevada em todo o planeta e que é evidentemente um risco para a saúde humana.\\

Contudo, o tratamento e diagnóstico vem sendo cada vez mais eficiente e facilitado por uso de tecnologias. Uma dessas tecnologias é a \textit{Machine Learning}, que é uma técnica de aprendizado supervisionado que permite ao usuário aplicar uma série de regras para identificar padrões e entender o comportamento de um conjunto de dados.\\

Nesse contexto, a taxa de sobrevida é uma estimativa utilizada que indica a probabilidade de que um paciente seja recuperado após um diagnóstico a partir de um histórico de dados de outros pacientes que também tiveram a mesma doença, com características semelhantes.\\

Então, com esse trabalho se discute o mais adequado modelo de classificação para prever a sobrevida de paciente com câncer de mama após 5 anos. Utiliza-se o conjunto de dados que contém casos de um estudo realizado entre 1958 e 1970 no Hospitaln Billings da Universidade de Chicago, acerca da sobrevivência de 306 pacientes que se submeteram a cirurgia para câncer de mama.

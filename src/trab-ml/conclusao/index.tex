
Ao analisar os cinco modelos propostos, observa-se que todos eles apresentam uma acurácia acima de $50\%$. Num caso como tal, por se tratar de um problema que envolve a saúde e a condição de vida de pacientes com cancer, quanto menor a quatidade de erros e maior a acurácia, melhor.
\\

Isto posto, Ao observar que o modelo KNN não somente apresentou maior acurácia (ACC), como também melhor predição de VP e todas as outras métricas melhores que os outros, pode-se ecolhê-lo como o modelo mais indicado para a análise.\\

Observa-se também que a análise inicial com os conjuntos de treinamento não foram contempladas, já que o modelo NB era o mais indicado a primeira vista.
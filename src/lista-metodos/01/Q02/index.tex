
\noindent \textcolor{COLOR1}{Questão 02)} Considere as seguintes matrizes:
\\
\newcommand{\mA}{
    \begin{bmatrix}
        2 & 0 \\
        6 & 7
    \end{bmatrix}}
\newcommand{\mB}{
    \begin{bmatrix}
        0 & 4  \\
        2 & -8
    \end{bmatrix}}
\newcommand{\mC}{
    \begin{bmatrix}
        -6 & 9  & -7 \\
        7  & -3 & -2
    \end{bmatrix}}
\newcommand{\mD}{
    \begin{bmatrix}
        -6 & 4 & 0 \\
        1  & 1 & 4 \\
        -6 & 0 & 6
    \end{bmatrix}
}
\newcommand{\mE}{
    \begin{bmatrix}
        6  & 9 & -9 \\
        -1 & 0 & -4 \\
        -6 & 0 & -1
    \end{bmatrix}
}

\[
    A =
    \mA,\hspace{1em}
    B =
    \mB,\hspace{1em}
    C =
    \mC
\]
\newline

\[
    D =
    \mD,\hspace{1em}
    E =
    \mE
\]
\\

Efetue as seguintes operações ou justifique porque elas não podem ser realizadas:
\\

% Respostas
a) $AB-BA$
\\

\[
    AB =
    \begin{bmatrix}
        0  & 8   \\
        14 & -32
    \end{bmatrix},\hspace{1em}
    BA =
    \begin{bmatrix}
        24  & 28  \\
        -44 & -56
    \end{bmatrix}
\]
\\

Assim sendo:
\\

\begin{equation}
    AB - BA =
    \begin{bmatrix}
        -24 & -20 \\
        58  & 24
    \end{bmatrix}
\end{equation}
\\

b) $2C-D$
\\

\[
    2C = 2 \cdot
    \mC =
    \begin{bmatrix}
        -12 & 18 & -14 \\
        14  & -6 & -4
    \end{bmatrix} -
    \mD
\]
\\


\textcolor{COLOR2}{Assim sendo, \textbf{\underline{não}} é possível efetuar essa operação, já que uma matrix é de dimensão $2\times3$ ($2C=M_{i,j}$) e outra, $3\times3$ ($D_{i, j}$). A subtração só é possível com $i$ e $j$ iguais em ambas as matrizes.}
\\

c) $(2D^T-3E^T)^T$
\\

\[
    2D^T = 2 \cdot
    \mD =
    \begin{bmatrix}
        -12 & 8 & 0  \\
        2   & 2 & 8  \\
        -12 & 0 & 12
    \end{bmatrix}^T =
    \begin{bmatrix}
        -12 & 2 & -12 \\
        8   & 2 & 0   \\
        0   & 8 & 12
    \end{bmatrix}
\]
\\

\[
    3E^T = 3 \cdot
    \mE =
    \begin{bmatrix}
        -18 & 27 & -27 \\
        -3  & 0  & -12 \\
        -18 & 0  & -3
    \end{bmatrix}^T =
    \begin{bmatrix}
        18  & -3  & -18 \\
        27  & 0   & 0   \\
        -27 & -12 & -3
    \end{bmatrix}
\]
\\

\begin{equation}
    (2D^T-3E^T)^T =
    \begin{bmatrix}
        -30 & -19 & 27 \\
        5   & 2   & 20 \\
        6   & 0   & 15
    \end{bmatrix}
\end{equation}
\\

d) $D^2-DE$
\\

\[
    D^2 =
    \mD^2 =
    \mD \cdot
    \mD =
    \begin{bmatrix}
        40  & -20 & 16 \\
        -29 & 5   & 28 \\
        0   & -24 & 36
    \end{bmatrix}
\]
\\

\[
    DE =
    \mD\cdot
    \mE
    =
    \begin{bmatrix}
        -40 & -54 & -38 \\
        19  & 9   & -17 \\
        -72 & -54 & -48
    \end{bmatrix}
\]
\\

Assim sendo:
\\

\begin{equation}
    D^2-DE =
    \begin{bmatrix}
        80  & 34 & -22 \\
        -10 & -4 & 45  \\
        72  & 30 & -12
    \end{bmatrix}
\end{equation}
\\

d) $(DC)^T$
\\

\[
    DC =
    \mD \cdot
    \mC
\]
\\


\textcolor{COLOR2}{\textbf{\underline{Não}} é possível efetuar essa operação, já que uma matrix é de dimensão $3\times3$ ($D_{i,j}$) e outra, $2\times3$ ($C_{i, j}$) e essa multiplicação não é válida porque em $D$, $j=3$ e em $C$, $i=2$. Para ser compatível esses valores devem ser iguais.}
\\

\noindent \textcolor{COLOR1}{Questão 04)} Considere a matriz $A = \begin{bmatrix}
        1 & 1 & 0  & 2 \\
        1 & 3 & 1  & 2 \\
        1 & 2 & -1 & 1 \\
        5 & 9 & 0  & 6
    \end{bmatrix}$ e uma matriz $B$ de tamanho $4\times 4$ tal que $det(B)=-3$. Determine:
\\

a-1) $det(A)$. Usando eliminação de Gauss-jordan para encontrar matriz triangular superior.
\\

\[
    \begin{bmatrix}
        1 & 1 & 0  & 2 \\
        1 & 3 & 1  & 2 \\
        1 & 2 & -1 & 1 \\
        5 & 9 & 0  & 6
    \end{bmatrix}
    R_2 - R_1 \to R_2
    \begin{bmatrix}
        1 & 1 & 0  & 2 \\
        0 & 2 & 1  & 0 \\
        1 & 2 & -1 & 1 \\
        5 & 9 & 0  & 6
    \end{bmatrix}
    R_3 - R_1 \to  R_3
    \begin{bmatrix}
        1 & 1 & 0  & 2  \\
        0 & 2 & 1  & 0  \\
        0 & 1 & -1 & -1 \\
        5 & 9 & 0  & 6
    \end{bmatrix}
\]
\[
    R_4 - 5R_1\to  R_4
    \begin{bmatrix}
        1 & 1 & 0  & 2  \\
        0 & 2 & 1  & 0  \\
        0 & 1 & -1 & -1 \\
        0 & 4 & 0  & -4
    \end{bmatrix}
    R_3 - \frac{1}{2} R_2\to  R_3
    \begin{bmatrix}
        1 & 1 & 0             & 2  \\
        0 & 2 & 1             & 0  \\
        0 & 0 & \sfrac{-3}{2} & -1 \\
        0 & 4 & 0             & -4
    \end{bmatrix}
    R_4 - 2R_2 \to R_4
\]
\\
\[
    \begin{bmatrix}
        1 & 1 & 0             & 2  \\
        0 & 2 & 1             & 0  \\
        0 & 0 & \sfrac{-3}{2} & -1 \\
        0 & 0 & -2            & -4
    \end{bmatrix}
    R_4 - \frac{4}{3}R_3 \to R_4
    \begin{bmatrix}
        1 & 1 & 0             & 2             \\
        0 & 2 & 1             & 0             \\
        0 & 0 & \sfrac{-3}{2} & -1            \\
        0 & 0 & 0             & \sfrac{-8}{3}
    \end{bmatrix}
\]
\\

Assim sendo, obtida a matriz triangular, temos que o determinante de A será:
\\

\begin{equation}
    \left|\begin{matrix}
        1 & 1 & 0  & 2 \\
        1 & 3 & 1  & 2 \\
        1 & 2 & -1 & 1 \\
        5 & 9 & 0  & 6
    \end{matrix}\right| =
    \left|\begin{matrix}
        1 & 1 & 0             & 2             \\
        0 & 2 & 1             & 0             \\
        0 & 0 & \sfrac{-3}{2} & -1            \\
        0 & 0 & 0             & \sfrac{-8}{3}
    \end{matrix}\right|= 1\times 2 \times \frac{-3}{2} \times \frac{-8}{3} = 8
\end{equation}
\\

a-2) $det(A^2)$.
\\

Como sabemos que $det(A) = 8$, temos que \textcolor{COLOR2}{$det(A^2) = det(A)\times det(A) = 64$}.\\

a-3) $det(B^3)$.
\\

Como sabemos que $det(B) = -3$, temos que \textcolor{COLOR2}{$det(B^2) = det(B)\times det(B) \times det(B) = -27$}.\\

a-4) $det(3B^{-1})$.
\\

Como sabemos que $det(B) = -3$, temos que \textcolor{COLOR2}{$det(3B^{-1}) = 3^4\times \frac{1}{det(B)} = -27$}.\\

a-5) $det(\frac{1}{4}A^T)$.
\\

Como sabemos que $det(A) =det(A^T) = 8$, temos que \textcolor{COLOR2}{$det(\frac{1}{4}A^T) ={\frac{1}{32}}^4 \times det(A) = \frac{1}{32}$}.\\

a-6) $det(A^TB^{-1})$.
\\

Como sabemos que $det(A) = 8$ e $det(B) = -3$, temos que \textcolor{COLOR2}{$det(A^TB^{-1})=8\times -\frac{1}{3} = -\frac{8}{3}$}.\\

b-1)$A^{-1}$. Usando eliminação de Gauss-jordan para encontrar matriz identidade.
\\

\[
    \begin{bmatrix}[cccc|cccc]
        1 & 1 & 0  & 2 & 1 & 0 & 0 & 0 \\
        1 & 3 & 1  & 2 & 0 & 1 & 0 & 0 \\
        1 & 2 & -1 & 1 & 0 & 0 & 1 & 0 \\
        5 & 9 & 0  & 6 & 0 & 0 & 0 & 1
    \end{bmatrix}
    R_2 - R_1\to R_2
    \begin{bmatrix}[cccc|cccc]
        1 & 1 & 0  & 2 & 1  & 0 & 0 & 0 \\
        0 & 2 & 1  & 0 & -1 & 1 & 0 & 0 \\
        1 & 2 & -1 & 1 & 0  & 0 & 1 & 0 \\
        5 & 9 & 0  & 6 & 0  & 0 & 0 & 1
    \end{bmatrix}
\]
\\
\[
    R_3 - R_1\to R_3
    \begin{bmatrix}[cccc|cccc]
        1 & 1 & 0  & 2  & 1  & 0 & 0 & 0 \\
        0 & 2 & 1  & 0  & -1 & 1 & 0 & 0 \\
        0 & 1 & -1 & -1 & -1 & 0 & 1 & 0 \\
        5 & 9 & 0  & 6  & 0  & 0 & 0 & 1
    \end{bmatrix}
    R_4 - 5R_1\to R_4
    \begin{bmatrix}[cccc|cccc]
        1 & 1 & 0  & 2  & 1  & 0 & 0 & 0 \\
        0 & 2 & 1  & 0  & -1 & 1 & 0 & 0 \\
        0 & 1 & -1 & -1 & -1 & 0 & 1 & 0 \\
        0 & 4 & 0  & -4 & -5 & 0 & 0 & 1
    \end{bmatrix}
\]
\\
\[
    \frac{R_2}{2}\to R_2
    \begin{bmatrix}[cccc|cccc]
        1 & 1 & 0   & 2  & 1    & 0   & 0 & 0 \\
        0 & 1 & 0,5 & 0  & -0,5 & 0,5 & 0 & 0 \\
        0 & 1 & -1  & -1 & -1   & 0   & 1 & 0 \\
        0 & 4 & 0   & -4 & -5   & 0   & 0 & 1
    \end{bmatrix}
    R_3 - R_2\to R_3
\]
\\
\[
    \begin{bmatrix}[cccc|cccc]
        1 & 1 & 0    & 2  & 1    & 0    & 0 & 0 \\
        0 & 1 & 0,5  & 0  & -0,5 & 0,5  & 0 & 0 \\
        0 & 0 & -1,5 & -1 & -0,5 & -0,5 & 1 & 0 \\
        0 & 4 & 0    & -4 & -5   & 0    & 0 & 1
    \end{bmatrix}
\]
\\
\[
    R_4 - 4R_2\to R4
    \begin{bmatrix}[cccc|cccc]
        1 & 1 & 0    & 2  & 1    & 0    & 0 & 0 \\
        0 & 1 & 0,5  & 0  & -0,5 & 0,5  & 0 & 0 \\
        0 & 0 & -1,5 & -1 & -0,5 & -0,5 & 1 & 0 \\
        0 & 0 & -2   & -4 & -3   & -2   & 0 & 1
    \end{bmatrix}
    -\frac{R_3}{1,5}\to R_3
\]
\\
\[
    \begin{bmatrix}[cccc|cccc]
        1 & 1 & 0   & 2      & 1      & 0      & 0       & 0 \\
        0 & 1 & 0,5 & 0      & -0,5   & 0,5    & 0       & 0 \\
        0 & 0 & 1   & 0,6667 & 0,3333 & 0,3333 & -0,6667 & 0 \\
        0 & 0 & -2  & -4     & -3     & -2     & 0       & 1
    \end{bmatrix}
    R_4 - (-2)R_3\to R_4
\]
\\
\[
    \begin{bmatrix}[cccc|cccc]
        1 & 1 & 0   & 2       & 1       & 0       & 0       & 0 \\
        0 & 1 & 0,5 & 0       & -0,5    & 0,5     & 0       & 0 \\
        0 & 0 & 1   & 0,6667  & 0,3333  & 0,3333  & -0,6667 & 0 \\
        0 & 0 & 0   & -2,6667 & -2,3333 & -1,3333 & -1,3333 & 1
    \end{bmatrix}
    -\frac{R_4}{2,6667}\to R_4
\]
\\
\[
    \begin{bmatrix}[cccc|cccc]
        1 & 1 & 0   & 2      & 1      & 0      & 0       & 0      \\
        0 & 1 & 0,5 & 0      & -0,5   & 0,5    & 0       & 0      \\
        0 & 0 & 1   & 0,6667 & 0,3333 & 0,3333 & -0,6667 & 0      \\
        0 & 0 & 0   & 1      & 0,875  & 0,5    & 0,5     & -0,375
    \end{bmatrix}
    R_3 - 0,6667R_4 \to R_3
\]
\\
\[
    \begin{bmatrix}[cccc|cccc]
        1 & 1 & 0   & 2 & 1     & 0   & 0   & 0      \\
        0 & 1 & 0,5 & 0 & -0,5  & 0,5 & 0   & 0      \\
        0 & 0 & 1   & 0 & -0,25 & 0   & -1  & 0,25   \\
        0 & 0 & 0   & 1 & 0,875 & 0,5 & 0,5 & -0,375
    \end{bmatrix}
    R_1 - 2R_4 \to R_1
\]
\\
\[
    \begin{bmatrix}[cccc|cccc]
        1 & 1 & 0   & 0 & -0,75 & -1  & -1  & 0,75   \\
        0 & 1 & 0,5 & 0 & -0,5  & 0,5 & 0   & 0      \\
        0 & 0 & 1   & 0 & -0,25 & 0   & -1  & 0,25   \\
        0 & 0 & 0   & 1 & 0,875 & 0,5 & 0,5 & -0,375
    \end{bmatrix}
    R_2 - 0,5R_3\to R_2
\]
\\
\[
    \begin{bmatrix}[cccc|cccc]
        1 & 1 & 0 & 0 & -0,75  & -1  & -1  & 0,75   \\
        0 & 1 & 0 & 0 & -0,375 & 0,5 & 0,5 & -0,125 \\
        0 & 0 & 1 & 0 & -0,25  & 0   & -1  & 0,25   \\
        0 & 0 & 0 & 1 & 0,875  & 0,5 & 0,5 & -0,375
    \end{bmatrix}
    R_1 - R_2\to R_1
\]
\\
\begin{equation}
    \begin{bmatrix}[cccc|cccc]
        1 & 0 & 0 & 0 & -0,375 & -1,5 & -1,5 & 0,875  \\
        0 & 1 & 0 & 0 & -0,375 & 0,5  & 0,5  & -0,125 \\
        0 & 0 & 1 & 0 & -0,25  & 0    & -1   & 0,25   \\
        0 & 0 & 0 & 1 & 0,875  & 0,5  & 0,5  & -0,375
    \end{bmatrix} =
    \begin{bmatrix}
        -0,375 & -1,5 & -1,5 & 0,875  \\
        -0,375 & 0,5  & 0,5  & -0,125 \\
        -0,25  & 0    & -1   & 0,25   \\
        0,875  & 0,5  & 0,5  & -0,375
    \end{bmatrix}
\end{equation}
\\

b-2) $(\frac{1}{2}A^T)^{-1}$. Isso é o mesmo que $2(A^{-1})^{T}$, logo:
\\

\begin{equation}
    (A^{-1})^T= 2\times
    \begin{bmatrix}
        -0,375 & -0,375 & -0,25 & 0,875  \\
        -1,5   & 0,5    & 0     & 0,5    \\
        -1,5   & 0,5    & -1    & 0,5    \\
        0,875  & -0,125 & 0,25  & -0,375
    \end{bmatrix} =
    \begin{bmatrix}
        -0,75 & -0,75 & -0,5 & 1,75  \\
        -3    & 1     & 0    & 1     \\
        -3    & 1     & -2   & 1     \\
        1,75  & -0,25 & 0,5  & -0,75
    \end{bmatrix}
\end{equation}
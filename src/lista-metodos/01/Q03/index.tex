\newcommand{\mAA}{
    \begin{bmatrix}
        a_{11} & a_{12} \\
        a_{21} & a_{22}
    \end{bmatrix}
}
\newcommand{\mBB}{
    \begin{bmatrix}
        b_{11} & b_{12} \\
        b_{21} & b_{22}
    \end{bmatrix}
}
\newcommand{\mM}{
    \begin{bmatrix}
        0  & 1 \\
        -1 & 0
    \end{bmatrix}
}

\noindent \textcolor{COLOR1}{Questão 03)} Mostre que se $A$ e $B$ são matrizes que comutam com a matriz $M=\mM$.
\\

Temos a matriz $A = \mAA$ e a matriz $B = \mBB$. Assim sendo:
\\

\[
    AM = \mAA \cdot \mM =
    \begin{bmatrix}
        -a_{12} & a_{11} \\
        -a_{22} & a_{21}
    \end{bmatrix}
\]


\[
    MA = \mM \cdot \mAA =
    \begin{bmatrix}
        a_{21}  & a_{22}  \\
        -a_{11} & -a_{12}
    \end{bmatrix}
\]
\\

A partir disso, podemos encontrar:
\\

\[
    \begin{cases}
        -a_{12} = a_{21}  \\
        a_{11} = a_{22}   \\
        -a_{22} = -a_{11} \\
        a_{21} = -a_{12}
    \end{cases}
\]
\\

Pode-se simplificar para a condição de que $a_{11} = a_{22} = r$, $a_{12} = s$ e $a_{21} = -s$. Similarmente, para a matriz $B$, temos $b_{11} = b_{22} = x$, $b_{12} = y$ e $b_{21} = -y$:
\\

\[
    A =
    \begin{bmatrix}
        r  & s \\
        -s & r
    \end{bmatrix},\hspace{1em}
    B =
    \begin{bmatrix}
        x  & y \\
        -y & x
    \end{bmatrix}
\]
\\

multiplicando-as, encontramos:
\\

\begin{equation}\label{eq:eq01}
    AB =
    \begin{bmatrix}
        rx-sy  & ry+sx  \\
        -sx-ry & -sy+rx
    \end{bmatrix},\hspace{1em}
    BA =
    \begin{bmatrix}
        rx-sy  & sx+ry  \\
        -ry-sx & -sy+rx
    \end{bmatrix}
\end{equation}
\\

\textcolor{COLOR2}{A partir da Equação \ref{eq:eq01} é possivel chegar a conclusão de que $AB = BA = M$.}
\\

\noindent \textcolor{COLOR1}{Questão 05)} Exercícios com python.
\\

a) Responda a questão 2 e a questão 4(b) usando Python. Especifique os comandos utilizados em cada uma e as respostas (ou os erros) obtidas\footnote{Utilizou-se a \textit{library} Pandas para cálculo de todas as operações}.
\\

2-a) $AB-BA$
\\

\begin{lstlisting}
import pandas as pd

# Cria as matrizes
A = pd.DataFrame([[2, 0], [6, 7]])
B = pd.DataFrame([[0, 4], [2, -8]])

# Calcula AB-BA
M=A.dot(B) - B.dot(A)

# Imprime valor
print(M)

#     0   1
# 0 -24 -20
# 1  58  24
\end{lstlisting}


2-b) $2C-D$. Efetua-se a subtração apenas das linhas que são possíveis retornando NaN (not a number) nas que não são possíveis.
\\

\begin{lstlisting}
import pandas as pd

C = pd.DataFrame([[-6, 9, -7], [7, -3, -2]])
D = pd.DataFrame([[-6, 4, 0], [1, 1, 4], [-6, 0, 6]])

# Calcula 2C-D (calculo pode ser feito diretamente)
M = 2*C-D

# Imprime valor
print(M)

#       0     1     2
# 0  -6.0  14.0 -14.0
# 1  13.0  -7.0  -8.0
# 2   NaN   NaN   NaN
\end{lstlisting}

2-c) $(2D^T-3E^T)^T$.
\\

\begin{lstlisting}
import pandas as pd

# Cria as matrizes
D = pd.DataFrame([[-6, 4, 0], [1, 1, 4], [-6, 0, 6]])
E = pd.DataFrame([[6, 9, -9], [-1, 0, -4], [-6, 0, -1]])
    
# Calcula (2D^T-3E^T)
M = 2*D.T-3*E.T
# Calcula a transposta de M
MT = M.T
# Imprime valor
print(MT)
    
#     0   1   2
# 0 -30 -19  27
# 1   5   2  20
# 2   6   0  15
\end{lstlisting}


2-d) $D^2-DE$.
\\

\begin{lstlisting}
import pandas as pd

# Cria as matrizes
D = pd.DataFrame([[-6, 4, 0], [1, 1, 4], [-6, 0, 6]])
E = pd.DataFrame([[6, 9, -9], [-1, 0, -4], [-6, 0, -1]])
    
# Calcula D^2-DE
M = D.dot(D)-D.dot(E)
# Imprime valor
print(M)
    
#     0   1   2
# 0  80  34 -22
# 1 -10  -4  45
# 2  72  30 -12
\end{lstlisting}


2-d) $(DC)^T$. Erro quando calcula a multiplicação entre D e C
\\

\begin{lstlisting}
import pandas as pd

# Cria as matrizes
A = pd.DataFrame([[2, 0], [6, 7]])
B = pd.DataFrame([[0, 4], [2, -8]])
C = pd.DataFrame([[-6, 9, -7], [7, -3, -2]])
D = pd.DataFrame([[-6, 4, 0], [1, 1, 4], [-6, 0, 6]])
E = pd.DataFrame([[6, 9, -9], [-1, 0, -4], [-6, 0, -1]])

# Calcula DC
M = D.dot(C)
# Calcula M^T
MT = M.T
# Imprime valor
print(MT)

# Error in M = D.dot(C)
# raise ValueError("matrices are not aligned")
# ValueError: matrices are not aligned
\end{lstlisting}


b) Realize o seguinte experimento para tentar ter uma ideia de quão comum é encontrar matrizes cujo produto comuta. Em Python, digite as seguintes linhas de comando:
\\

\begin{lstlisting}
import numpy as np
c=0

for i in range(10000):
    A=np.random.randint(-5,5,size=(3,3))
    B=np.random.randint(-5,5,size=(3,3))
    C=A@B==B@A
    if C.all():
        c=c+1
print(c)    
\end{lstlisting}


Na primeira linha é o \textit{import} da \textit{library}, na segunda atribuição de valor a uma variável inteira, na terceira o ínicio do loop até 10000, na quarta e quinta duas matrizes randômicas de tamanho $3\times 3$ para as variáveis A e B, na sexta a comparação de igualdade entre os produtos A@B e B@A (verificando a comutação), na sétima a condição de que se ambos forem iguais a variável C é incrementada em 1, na oitava linha o valor de c é impresso.
\\

\textcolor{COLOR2}{A variável c poderá ter valores entre 0 e 10000, pois o loop é executado 10000 vezes, dependendo apenas se o produto A@B é igual a B@A ou não. As matrizes A e B são geradas randômicamente e podem ter valores entre -5 e 5 e a comutação pode não ocorrer em todos os casos.}
\\

c) Realize o seguinte experimento para tentar ter uma ideia de quão comum é encontrar matrizes invertíveis. Em Python, digite as seguintes linhas de comando:
\\

\begin{lstlisting}
import numpy as np
c=0
for i in range(10000):
    A=np.random.randint(-5,5,size=(4,4))
    if np.linalg.det(A) != 0:
        c+=1

print(c)
\end{lstlisting}

Na primeira linha é o \textit{import} da \textit{library}, na segunda atribuição de valor a uma variável inteira, na terceira o ínicio do loop até 10000, na quarta e quinta uma matriz randômica de tamanho $4\times 4$ para a variável A, na sexta a condição de que se o determinante for diferente de 0 a variável c é incrementada em 1, na sétima linha o valor de c é impresso.
\\

\textcolor{COLOR2}{A variável c poderá ter valores entre 0 e 10000, pois o loop é executado 10000 vezes, dependendo apenas se o determinante for diferente de 0 ou não. As matrizes A são geradas randômicamente e podem ter valores entre -5 e 5.}
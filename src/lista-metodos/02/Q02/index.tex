
\noindent \textcolor{COLOR1}{Questão 02)} Usando o método de Gauss-Jordan resolva o sistema de equações lineares.
\\

\[
    \begin{cases}
        x_1 + 2x_2 - 3x_4 + x_5 = 2              \\
        x_1 + 2x_2 + x_3 - 3x_4 + x_5 + 2x_6 = 3 \\
        x_1 + 2x_2 - 3x_4 + 2x_5 + x_6 = 4       \\
        3x_1 + 6x_2 + x_3 - 9x_4 + 4x_5 + 3x_6 = 9
    \end{cases}
\]
\\


No processo de eliminação gaussiana, encontra-se a matriz escalonada a partir da matriz aumentada:
\\

\[
    \begin{bmatrix}[cccccc|c]
        1 & 2 & 0 & -3 & 1 & 0 & 2 \\
        1 & 2 & 1 & -3 & 1 & 2 & 3 \\
        1 & 2 & 0 & -3 & 2 & 1 & 4 \\
        3 & 6 & 1 & -9 & 4 & 3 & 9
    \end{bmatrix}
    R_2 - R_1 \to R_2
    \begin{bmatrix}[cccccc|c]
        1 & 2 & 0 & -3 & 1 & 0 & 2 \\
        0 & 0 & 1 & 0  & 0 & 2 & 1 \\
        1 & 2 & 0 & -3 & 2 & 1 & 4 \\
        3 & 6 & 1 & -9 & 4 & 3 & 9
    \end{bmatrix}
    R_3 - R_1 \to R_3
\]


\[
    \begin{bmatrix}[cccccc|c]
        1 & 2 & 0 & -3 & 1 & 0 & 2 \\
        0 & 0 & 1 & 0  & 0 & 2 & 1 \\
        0 & 0 & 0 & 0  & 1 & 1 & 2 \\
        3 & 6 & 1 & -9 & 4 & 3 & 9
    \end{bmatrix}
    R_4-3R_1 \to R_4
    \begin{bmatrix}[cccccc|c]
        1 & 2 & 0 & -3 & 1 & 0 & 2 \\
        0 & 0 & 1 & 0  & 0 & 2 & 1 \\
        0 & 0 & 0 & 0  & 1 & 1 & 2 \\
        0 & 0 & 1 & 0  & 1 & 3 & 3
    \end{bmatrix}
    R_4-R_2\to R_4
    \begin{bmatrix}[cccccc|c]
        1 & 2 & 0 & -3 & 1 & 0 & 2 \\
        0 & 0 & 1 & 0  & 0 & 2 & 1 \\
        0 & 0 & 0 & 0  & 1 & 1 & 2 \\
        0 & 0 & 0 & 0  & 1 & 1 & 2
    \end{bmatrix}
\]


\[
    -R_4-R_3\to R_4
    \begin{bmatrix}[cccccc|c]
        1 & 2 & 0 & -3 & 1 & 0 & 2 \\
        0 & 0 & 1 & 0  & 0 & 2 & 1 \\
        0 & 0 & 0 & 0  & 1 & 1 & 2 \\
        0 & 0 & 0 & 0  & 0 & 0 & 0
    \end{bmatrix}
    R_1-R_3\to R_1
    \begin{bmatrix}[cccccc|c]
        1 & 2 & 0 & -3 & 0 & -1 & 0 \\
        0 & 0 & 1 & 0  & 0 & 2  & 1 \\
        0 & 0 & 0 & 0  & 1 & 1  & 2 \\
        0 & 0 & 0 & 0  & 0 & 0  & 0
    \end{bmatrix}
\]
\\

Essa matriz escalonada resulta em um sistema de equações lineares:
\\

\[
    \begin{cases}
        x_1 +2*x_2  -3*x_4 -x_6  = 0 \\
        x_3  +2*x_6 = 1              \\
        x_5 +x_6 = 2
    \end{cases}
\]
\\

Pela matriz aumentada escalonada é possível notar que seu posto é igual a $p(M|B)=3$. Sabe-se também que o posto da matriz dos coeficientes é igual a $p(M)=3$. Com base nisso e tendo como número de variáveis $n=6$, podemos afirmar que o sistema é possível, mas indeterminado. Ou seja, apresenta infinitas soluções.\\

\begin{equation}
    \begin{cases}
        x_1=-2*x_2+3*x_4+x_6 \\
        x_2=x_2              \\
        x_3=1-2*x_6          \\
        x_4=x_4              \\
        x_5=2-x_6            \\
        x_6=x_6
    \end{cases}
\end{equation}


\noindent \textcolor{COLOR1}{Questão 03)} A Regra de Cramer é um método para determinar a solução de um sistema linear
$Ax = b$ de n equações e n incógnitas quando a matriz dos coeficientes A é invertível.
\\

\noindent\fbox{%
    \parbox{\textwidth}{%
        \textbf{Regra de Cramer}: Se $Ax = b$ é um sistema de n equações lineares em n incógnitas tal que $det(A)\neq 0$, então a única solução do sistema é dada por:
        \[
            x_1 = \frac{det(A_1)}{det(A)}, \hspace{0.4em}
            x_2 = \frac{det(A_2)}{det(A)}, \hspace{0.4em} \dots, \hspace{0.4em}
            x_n = \frac{det(A_n)}{det(A)}
        \]
        \\

        onde $A_j$ é a matriz obtida substituindo as entradas da $j-ésima$ coluna de $A$ pelas entradas da matriz $b$.
    }%
}
\\
\\

a) Mostre a Regra de Cramer no caso em que $A$ é uma matriz quadrada de ordem 2.
\\

Tem-se então:
\\


\[
    \begin{cases}
        a_1x_1 +a_2x_2  =  b_1 \\
        a_3x_1  +a_4x_2  =  b_2
    \end{cases}
\]
\\

Onde podemos obter a matriz dos coeficientes:
\[
    A=
    \begin{bmatrix}
        a_1 & a_2 \\
        a_3 & a_4
    \end{bmatrix}
\]
\\

o determinante de $A$ é igual a:
\\

\[
    \det(A) =
    \left|\begin{matrix}
        a_1 & a_2 \\
        a_3 & a_4
    \end{matrix}\right|
    =-a_2\cdot a_3+a_1\cdot a_4
\]
\\

Os determinantes de $A_1$ e $A_2$ são iguais a:
\\

\[
    \det(A_1) =
    \left|\begin{matrix}
        b_1 & a_2 \\
        b_2 & a_4
    \end{matrix}\right|
    =a_4\cdot b_1-a_2\cdot b_2
\]

\[
    \det(A_2) =
    \left|\begin{matrix}
        a_1 & b_1 \\
        a_3 & b_2
    \end{matrix}\right|
    =-a_3\cdot b_1+a_1\cdot b_2
\]
\\

Obtem-se então a solução do sistema:
\\

\[
    \begin{cases}
        x_1 = \frac{-a_4\cdot b_1+a_2\cdot b_2}{a_2\cdot a_3-a_1\cdot a_4} \\
        x_2 = \frac{a_3\cdot b_1-a_1\cdot b_2}{a_2\cdot a_3-a_1\cdot a_4}
    \end{cases}
\]
\\

b) Use a Regra de Cramer para resolver $x'$ e $y'$ em termos de $x$ e $y$
\\


\[
    \begin{cases}
        x= \frac{3}{5}x' -\frac{4}{5}y' \\
        y = \frac{4}{5}x'  +\frac{3}{5}y'
    \end{cases}
\]
\\

A matriz dos coeficientes:
\\

\[
    A=
    \begin{bmatrix}
        \frac{3}{5} & \frac{-4}{5} \\
        \frac{4}{5} & \frac{3}{5}
    \end{bmatrix}
\]

\[
    det(A)=1
\]
\\

\[
    \det(A_1) =
    \left|\begin{matrix}
        x & \sfrac{-4}{5} \\
        y & \sfrac{3}{5}
    \end{matrix}\right|
    = \frac{3x+4y}{5}
\]
\\

\[
    \det(A_2) =
    \left|\begin{matrix}
        \sfrac{3}{5} & x \\
        \sfrac{4}{5} & y
    \end{matrix}\right|
    = \frac{-4x+3y}{5}
\]
\\

Assim sendo, a solução do sistema é:
\\

\begin{equation}
    \begin{aligned}
         & x' = \sfrac{\frac{3x+4y}{5}}{1} = \frac{3x+4y}{5}   \\
         & y' = \sfrac{\frac{-4x+3y}{5}}{1} = \frac{-4x+3y}{5}
    \end{aligned}
\end{equation}



\noindent \textcolor{COLOR1}{Questão 01)} Determine as condições que as constantes $b_1$, $b_2$, $b_3$ devem satisfazer para garantir a consistência do sistema linear.
\\

\[
    \begin{cases}
        x - 2y + 5z = b_1  \\
        4x - 5y + 8z = b_2 \\
        -3x + 3y - 3z = b_3
    \end{cases}
\]
\\

Duas perguntas têm que ser feitas para essa verificação:
\\

\begin{enumerate}[label=(\alph*)]
    \item É um sistema possível de ser resolvido?
    \item É um sistema possível determinada/única ou indeterminadas soluções?
\end{enumerate}

Para responder essas questões, primeiramente devemos encontrar a \textbf{matriz escalonada}, por meio de operações elementares na matriz dos coeficientes.

Usando eliminação de gauss, tem-se:
\\

\[
    M=
    \begin{bmatrix}
        1  & -2 & 5  \\
        4  & -5 & 8  \\
        -3 & 3  & -3
    \end{bmatrix}
    R_2 - 4R_1 \to R_2
    \begin{bmatrix}
        1  & -2 & 5   \\
        0  & 3  & -12 \\
        -3 & 3  & -3
    \end{bmatrix}
    R_3 - (-3)R_1 \to  R_3
    \begin{bmatrix}
        1 & -2 & 5   \\
        0 & 3  & -12 \\
        0 & -3 & 12
    \end{bmatrix}
\]
\\

\[
    R_3 - (-1) R_2 \to R_3
    \begin{bmatrix}
        1 & -2 & 5   \\
        0 & 3  & -12 \\
        0 & 0  & 0
    \end{bmatrix}
\]
\\

A partir disso, obtemos o seu posto, isto é, o número de linhas não-nulas quando a matriz é escalonada $p(M) = 2$. Ainda, para a matriz aumentada teremos que \textbf{qualquer número real para os valores de $b_1$, $b_2$ e $b_3$ satizfazem a condição de que seu posto seja também $p(M|B) = 3$}. Assim sendo, o sistema é um sistema possível, já que temos uma solução única, ou seja $p(M|B) = p(M)$. Além disso, é também determinado, com $p(M)=n$, sendo $n$ o número de variáveis. Na condição de que o sistema apresentasse $p(M)=p(M|B)<n$ ele seria consistente e indeterminado, ou seja, haveria infinitas soluções. No caso de $p(M|B)>p(M)$ o sistema seria inconsistente e não seria possível de ser resolvido.
\\
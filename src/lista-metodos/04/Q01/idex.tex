
\noindent \textcolor{COLOR1}{Questão 01)} Considere o espaço vetorial $M_2(\mathbb{R}) $ com o produto interno dado por $\langle U, V\rangle = tr(U^T V )$. Determine se as seguintes funções são transformações lineares ou não. Justifique!
\\

a) $T: M_2(\mathbb{R})\to \mathbb{R}$ com $T(U)=\|U\|$:\\

Devemos considerar a soma e a multiplicação por uma constante e $\|U\|=\sqrt{tr(U^TU)}$. Assim sendo, tendo a matriz $U$:\\

\[
    U=\begin{bmatrix}
        u_{11} & u_{12} \\
        u_{21} & u_{22}
    \end{bmatrix}, \quad
    U^T=\begin{bmatrix}
        u_{11} & u_{21} \\
        u_{12} & u_{22}
    \end{bmatrix}
\]
\\

Assim:\\

\[
    \begin{split}
        &U^TU=\begin{bmatrix}
            u_{11}u_{11} & u_{21}u_{21} & u_{11}u_{12} & u_{21}u_{22} \\
            u_{12}u_{11} & u_{22}u_{21} & u_{12}u_{12} & u_{22}u_{22}
        \end{bmatrix}\\ \\
        &tr(U^TU)=u_{11}^2 + u_{21}^2 + u_{12}^2 + u_{22}^2\\
        &T(U)=\|U\| = \sqrt{u_{11}^2 + u_{21}^2 + u_{12}^2 + u_{22}^2}
    \end{split}
\]\\

1) Para a multiplicação por uma contante $\alpha$:
\\

\[
    \begin{split}
        &T(\alpha U) = \sqrt{\alpha^2(u_{11}^2 + u_{21}^2 + u_{12}^2 + u_{22}^2)}\\
        &T(\alpha U) = \alpha\sqrt{u_{11}^2 + u_{21}^2 + u_{12}^2 + u_{22}^2}\\
        &T(\alpha U) = \alpha T(U)
    \end{split}
\]\\

1) Para a soma com uma matriz $V$:\\

\[
    \begin{split}
        &T(U+V) = \sqrt{(u_{11}+v_{11})^2+(u_{21}+v_{21})^2+(u_{12}+v_{12})^2+(u_{22}+v_{22})^2}\\
        &T(U) + T(V)= \sqrt{u_{11}^2 + u_{21}^2 + u_{12}^2 + u_{22}^2} + \sqrt{v_{11}^2 + v_{21}^2 + v_{12}^2 + v_{22}^2}\\
        &T(U+V)\neq T(U)+T(V)
    \end{split}
\]
\\

Sendo assim, \textcolor{COLOR2}{não é uma transformação linear}, já que $T(U+v)=0$ e $T(U)+T(V) = 2T(U)$, sendo $V=-U$.\\

b) Se $B$ é uma matriz fixa de tamanho $2\times 3$, $T: M_2(\mathbb{R})\to M_{2\times 3}(\mathbb{R})$ com $T(U)=UB$:\\

Temos as três matrizes:\\

\[
    B=\begin{bmatrix}
        b_{11} & b_{12} & b_{13} \\
        b_{21} & b_{22} & b_{23}
    \end{bmatrix}, \quad
    U=\begin{bmatrix}
        u_{11} & u_{12} \\
        u_{21} & u_{22}
    \end{bmatrix}, \quad
    V=\begin{bmatrix}
        v_{11} & v_{12} \\
        v_{21} & v_{22}
    \end{bmatrix}
\]
\\

Assim sendo, temos:\\

\[
    T(U) = \begin{bmatrix}
        u_{11}b_{11} + u_{12}b_{21} & u_{11}b_{12} + u_{12}b_{22} & u_{11}b_{13} + u_{12}b_{23} \\
        u_{21}b_{11} + u_{22}b_{21} & u_{21}b_{12} + u_{22}b_{22} & u_{21}b_{13} + u_{22}b_{23}
    \end{bmatrix}
\]
\\

1) Para a multiplicação por uma contante $\alpha$:
\\

\[
    \begin{split}
        &T(\alpha U) = \begin{bmatrix}
            \alpha (u_{11}b_{11} +  u_{12}b_{21}) & \alpha (u_{11}b_{12} + u_{12}b_{22}) & \alpha (u_{11}b_{13} +  u_{12}b_{23}) \\
            \alpha (u_{21}b_{11} +  u_{22}b_{21}) & \alpha (u_{21}b_{12} + u_{22}b_{22}) & \alpha (u_{21}b_{13} +  u_{22}b_{23})
        \end{bmatrix}\\
        &T(\alpha U) = \alpha\begin{bmatrix}
            u_{11}b_{11} + u_{12}b_{21} & u_{11}b_{12} + u_{12}b_{22} & u_{11}b_{13} + u_{12}b_{23} \\
            u_{21}b_{11} + u_{22}b_{21} & u_{21}b_{12} + u_{22}b_{22} & u_{21}b_{13} + u_{22}b_{23}
        \end{bmatrix}\\
        &T(\alpha U) = \alpha T(U)
    \end{split}
\]
\\

2) Para a adição de uma matriz $V$:\\

\[
    \begin{split}
        &T(U+V) = \begin{bmatrix}
            (u_{11}+v_{11})b_{11} + (u_{12}+v_{12})b_{21} & \cdots & (u_{11}+v_{11})b_{13} + (u_{12}+v_{12})b_{23} \\
            (u_{21}+v_{21})b_{11} + (u_{22}+v_{22})b_{21} & \cdots & (u_{21}+v_{21})b_{13} + (u_{22}+v_{22})b_{23}
        \end{bmatrix}\\
        &T(U+V) = \begin{bmatrix}
            (u_{11}b_{11}+u_{12}b_{21}) + (v_{11}b_{11}+v_{12}b_{21}) & \cdots & (u_{11}b_{13}+u_{12}b_{23}) + (v_{11}b_{13}+v_{12}b_{23}) \\
            (u_{21}b_{11}+u_{22}b_{21}) + (v_{21}b_{11}+v_{22}b_{21}) & \cdots & (u_{21}b_{13}+u_{22}b_{23}) + (v_{21}b_{13}+v_{22}b_{23})
        \end{bmatrix}\\
        &T(U+V)= T(U)+T(V)
    \end{split}
\]
\\

Sendo assim, \textcolor{COLOR2}{é uma transformação linear}, já que ambas as condições são satisfeitas.\\
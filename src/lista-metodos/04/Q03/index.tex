
\noindent \textcolor{COLOR1}{Questão 03)} Considere a matriz:
\\

\[A=
    \begin{bmatrix}
        7  & -2 & 0  \\
        -2 & 6  & -2 \\
        0  & -2 & 5
    \end{bmatrix}
\]\\

a) Determine os autovalores de $A$ uma base ortonormal do autoespaço associado ao autovalor dominante de $A$.\\

Sabemos que o valor de $A-\lambda I$ será:\\

\[
    A-\lambda I =
    \begin{bmatrix}
        7-\lambda & -2        & 0         \\
        -2        & 6-\lambda & -2        \\
        0         & -2        & 5-\lambda
    \end{bmatrix}
\]\\

Segunda etapa, encontra-se o determinante dessa matriz para saber os $\lambda_n$:\\

\[
    det(A-\lambda I) = (\lambda-3)(\lambda-6)(\lambda-9)
\]\\

Assim, os valores de $\lambda$ são $(\lambda_1, \lambda_2, \lambda_3)=(3,6,9)$ que são substituídos na matriz $A$ e resolvida como um sistema homogêneo:\\

Para $\lambda_2=9$, temos então:\\

\[
    \begin{bmatrix}
        7-9 & -2  & 0   \\
        -2  & 6-9 & -2  \\
        0   & -2  & 5-9
    \end{bmatrix} = \begin{bmatrix}
        -2 & -2 & 0  \\
        -2 & -3 & -2 \\
        0  & -2 & -4
    \end{bmatrix}
\]

\[
    \begin{bmatrix}
        -2 & -2 & 0  \\
        -2 & -3 & -2 \\
        0  & -2 & -4
    \end{bmatrix}\cdot\begin{Bmatrix}
        x_1 \\
        x_2 \\
        x_3
    \end{Bmatrix} = \begin{Bmatrix}
        0 \\
        0 \\
        0
    \end{Bmatrix}
\]\\

Resolvendo o sistema de equações lineares (a matriz aumentada) com eliminação de gauss, temos:\\

\[
    \begin{bmatrix}[ccc|c]
        1 & 0 & -2 & 0 \\
        0 & 1 & 2  & 0 \\
        0 & 0 & 0  & 0
    \end{bmatrix}
\]\\

O que significa que:\\

\[
    \begin{split}
        &x_1 = 2x_3\\
        &x_2 = -2x_3\\
        &x_3 = x_3
    \end{split}
\]
\\

A solução é dada como $\{x_3(2,-2, 1)\}$ o que ao tomar $x_3=1$, ter-se-á $v_3=(2,-2, 1)$. A norma de $v_3\to \|v_3\|=3$. Assim $(\sfrac{2}{3}, \sfrac{-2}{3}, \sfrac{1}{3})$.\\


b)Com ajuda do Python realize 5 iterações do método das potências aplicado à matriz $A$, começando com $x_0=(1,0,0)$ Compare a aproximação resultante com o valor exato do autovalor dominante e o autovetor unitário correspondente. Observação: Especifique os comandos utilizados em Python.


\begin{lstlisting}
import numpy as np
from math import sqrt

#Calcula a norma do vetor
def norm(vet):
    sum = 0
    for v in vet:
        sum += v[0]**2
    return sqrt(sum)

A = np.array([[7, -2, 0],
        [-2, 6, -2],
        [0, -2, 5]])

xi = np.array([[1], [0], [0]])
print(f"x0:\n{xi}\n")

for i in range(5):
    Ax = A @ xi #multiplicacao da matriz por um vetor i
    print(f"Ax{i+1}:\n{Ax}") #imprime o resultado da multiplicacao
    xi = Ax/norm(Ax) #divide o vetor por sua norma
    print(f"x{i+1}:\n{xi}") #imprime o resultado da divisao
    print(f"Autovalor na iteracao {i+1} e: {norm(Ax)}\n") #imprime 
    o autovalor

# x0:
# [[1]
#  [0]
#  [0]]

# Ax1:
# [[ 7]
#  [-2]
#  [ 0]]
# x1:
# [[ 0.96152395]
#  [-0.27472113]
#  [ 0.        ]]
# Autovalor na iteracao 1 e: 7.280109889280518

# Ax2:
# [[ 7.28010989]
#  [-3.57137466]
#  [ 0.54944226]]
# x2:
# [[ 0.89573556]
#  [-0.43941744]
#  [ 0.06760268]]
# Autovalor na iteracao 2 e: 8.12752137946034

# Ax3:
# [[ 7.14898378]
#  [-4.56318114]
#  [ 1.2168483 ]]
# x3:
# [[ 0.83437796]
#  [-0.53258167]
#  [ 0.14202178]]
# Autovalor na iteracao 3 e: 8.568040094092964

# Ax4:
# [[ 6.90580904]
#  [-5.14828952]
#  [ 1.77527225]]
# x4:
# [[ 0.78522432]
#  [-0.58538574]
#  [ 0.20185715]]
# Autovalor na iteracao 4 e: 8.794695848686912

# Ax5:
# [[ 6.6673417 ]
#  [-5.48647736]
#  [ 2.18005723]]
# x5:
# [[ 0.74867875]
#  [-0.61607897]
#  [ 0.24479959]]
# Autovalor na iteracao 5 e: 8.905477454667345
\end{lstlisting}

O autovalor dominante encontrado deu muito próximo do valor calculado manualmente. Ja o autovetor, seria necessaria mais iterações para melhorar sua precisão.\\



\noindent \textcolor{COLOR1}{Questão 02)} Considere a transformação linear $T: \mathbb{R}^2\to \mathbb{R}^2$ tal que $T(1,0)=(1,-1)$ e $T(0,1)=(1,1)$. Determine:\\

a) $T(2,3)$:\\

\[
    \begin{array}{l}
        T(2,3) = T(2,0) + T(0,3)   \\
        T(2,3) = 2T(1,0) + 3T(0,1) \\
        T(2,3) = 2(1,-1) + 3(1,1)  \\
        T(2,3) = (5,1)
    \end{array}
\]\\

b) O vetor $(x,y)\in \mathbb{R}^2 $ tal que $T(x,y)=\left(4,-2\right)$:\\
\\

\[
    \begin{array}{l}
        T(x,y) = (4,-2)           \\
        T(x,y) = 3(1,-1) + (1,1)  \\
        T(x,y) = 3T(1,0) + T(0,1) \\
        T(x,y) = T(3,1)           \\
        (x,y) = (3,1)
    \end{array}
\]
\\

c) O núcleo e a imagem de $T$:\\

Sabemos que $(x,y)$ é igual a:\\

\[
    \begin{array}{l}
        (x,y) = vT(0,1) + uT(1,0) \\
        (x,y) = v(1,1) + u(1,-1)
    \end{array}
\]
\\

com $x=u+v$ e $y=v-u$.\\

\[
    \begin{array}{l}
        u+v=v-u \\
        u=-u    \\
        2u=0    \\
        u=0     \\
    \end{array}
\]
\\

Assim sendo, $(0,0)=v(1,1)$, $v=0$ e o núcleo desse vetor $(0,0)$ será dado como:\\

\[
    \begin{split}
        &T(x,y) = yT(0,1) + xT(1,0) \\
        &T(x,y) = y(1,1) + x(1,-1) \\
        &T(x,y) = (y,y) + (x,-x) \\
        &T(x,y) = (y+x,y-x) \\
    \end{split}
\]\\

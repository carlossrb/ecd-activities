
\noindent \textcolor{COLOR1}{Questão 03)} Se $A$ é uma matriz arbitrária de tamanho $m\times n$, o espaço nulo de $A$ é definido como sendo o subespaço vetorial de $\mathbb{R}^n$ formado pelas soluções do sistema homogêneo $Ax = 0$. A dimensão do espaço nulo de $A$ é chamada de nulidade de $A.$
\\

\begin{enumerate}
    \item Se $A$ é a matriz
          \[A=
              \begin{bmatrix}
                  2 & 3  & -1 & 1 \\
                  3 & -1 & 1  & 2
              \end{bmatrix}
          \]
          Determine a nulidade de $A$ e uma base para o espaço nulo de $A$.
          \\

          Primeiramente, reduz-se a matriz aumentada do sistema homogêneo $Ax = 0$ a forma escalonada:
          \\

          \[A=
              \begin{bmatrix}[cccc|c]
                  2 & 3              & -1           & 1            & 0 \\
                  0 & \sfrac{-11}{2} & \sfrac{5}{2} & \sfrac{1}{2} & 0
              \end{bmatrix}
          \]
          \\

          Sabemos que o posto da matriz é dado pela quantidade de linhas não nulas da matriz escalonada. Isto é, tem-se um $p(A)=2$. A nulidade é dada pelo número de colunas da matriz $A$ escalonada subtraído pelo posto da matriz. Ou seja, $nulidade(A) = 4 - p(A) = 2$.
          \\

          Para encontrar uma base de $A$, resolve-se o sistema escalonado $A x = 0$ obtendo $x$:
          \\

          \[
              \begin{cases}
                  x_1=-\frac{2}{11}x_3-\frac{7}{11}x_4  \\
                  x_2 = \frac{5}{11}x_3+\frac{1}{11}x_4 \\
                  x_3 = x_3                             \\
                  x_4 = x_4
              \end{cases}
          \]
          \\

          e isso implica no mesmo que reorganizar em:
          \\

          \[
              x_3 \cdot
              \begin{bmatrix}
                  \sfrac{-2}{11} \\
                  \sfrac{5}{11}  \\
                  1              \\
                  0
              \end{bmatrix} +
              x_4 \cdot
              \begin{bmatrix}
                  \sfrac{-7}{11} \\
                  \sfrac{1}{11}  \\
                  0              \\
                  1
              \end{bmatrix}
          \]
          \\

          A base para o espaço nulo de $A$ é formada pelos vetores entre colchetes, ou seja, $\{(\sfrac{-2}{11}, \sfrac{5}{11},1,0),(\sfrac{-7}{11},\sfrac{1}{11},0,1)\}$\\

    \item Verifique que o vetor $v = (-10, 1, -1, 16)$ pertence ao espaçoo nulo de $A$ e determine as coordenadas deste vetor na base que você encontrou no item anterior.

\end{enumerate}
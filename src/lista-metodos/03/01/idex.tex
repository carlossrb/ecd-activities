
\noindent \textcolor{COLOR1}{Questão 01)} Lembre que uma matriz quadrada é dita triangular inferior quando todos os elementos acima da diagonal principal da matriz são nulos. Mostre que o conjunto $V_1\subset M_5 (\mathbb{R})$ das matrizes triangulares inferiores de ordem 5 é um subespaço vetorial de $M_5 (\mathbb{R})$.
\\

Para demonstrar que o conjunto $V_1\subset M_5 (\mathbb{R})$ é um subespaço vetorial de $M_5 (\mathbb{R})$, é necessário obedecer determinador requisitos. Tendo os atendido, então todos os axiomas de um espaço vetorial são obedecidos.

\begin{enumerate}
    \item $M_5$ deve necessariamente ser um espaço vetorial;
    \item Se $u$ e $v$ pertencem a $V_1$, então $u + v$ pertencem a $M_5$;
    \item Se $a$ for um escalar qualqur e $u$ pertence a $V_1$, então $au$ pertence a $M_5$.
\end{enumerate}

Para a primeira condição:\\
\[
    u + v =
    \begin{bmatrix}
        u_{11} & 0      & 0      & 0      & 0      \\
        u_{21} & u_{22} & 0      & 0      & 0      \\
        u_{31} & u_{32} & u_{33} & 0      & 0      \\
        u_{41} & u_{42} & u_{43} & u_{44} & 0      \\
        u_{51} & u_{52} & u_{53} & u_{54} & u_{55}
    \end{bmatrix} +
    \begin{bmatrix}
        v_{11} & 0      & 0      & 0      & 0      \\
        v_{21} & v_{22} & 0      & 0      & 0      \\
        v_{31} & v_{32} & v_{33} & 0      & 0      \\
        v_{41} & v_{42} & v_{43} & v_{44} & 0      \\
        v_{51} & v_{52} & v_{53} & v_{54} & v_{55}
    \end{bmatrix}=
\]
\\
\[
    u + v =
    \begin{bmatrix}
        u_{11}+v_{11} & 0             & 0             & 0             & 0             \\
        u_{21}+v_{21} & u_{22}+v_{22} & 0             & 0             & 0             \\
        u_{31}+v_{31} & u_{32}+v_{32} & u_{33}+v_{33} & 0             & 0             \\
        u_{41}+v_{41} & u_{42}+v_{42} & u_{43}+v_{43} & u_{44}+v_{44} & 0             \\
        u_{51}+v_{51} & u_{52}+v_{52} & u_{53}+v_{53} & u_{54}+v_{54} & u_{55}+v_{55}
    \end{bmatrix}
\]
\\

Para a segunda condição:
\\
\[
    au =
    \begin{bmatrix}
        au_{11} & 0       & 0       & 0       & 0       \\
        au_{21} & au_{22} & 0       & 0       & 0       \\
        au_{31} & au_{32} & au_{33} & 0       & 0       \\
        au_{41} & au_{42} & au_{43} & au_{44} & 0       \\
        au_{51} & au_{52} & au_{53} & au_{54} & au_{55}
    \end{bmatrix}
\]
\\

Para ambas as condições, a matriz resultando é uma matriz triangular inferior de dimensão $5\times 5$ na qual é possivel realizar todas as operações realizáveis em uma matriz de dimensão $5\times 5$ e que satisfazem os 10 axiomas definidores de um espaço vetorial.
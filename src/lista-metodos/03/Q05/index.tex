
\noindent \textcolor{COLOR1}{Questão 05)} Considere o espaço vetorial $M_{2\times3}\mathbb{R}$ com o produto interno $\langle U, V\rangle= tr(U^TV)$.
Aplique o processo de Gram-Schmidt para transformar a base
\\

\[
    U_1 = \begin{bmatrix}
        1 & 0 & 2 \\
        0 & 1 & 1
    \end{bmatrix}\hspace{0.5cm}
    U_2 = \begin{bmatrix}
        1 & 2 & 1 \\
        1 & 1 & 1
    \end{bmatrix}\hspace{0.5cm}
    U_3 = \begin{bmatrix}
        1 & 2 & 0 \\
        0 & 3 & 1
    \end{bmatrix}
\]
\[
    U_4 = \begin{bmatrix}
        1  & 0 & 1 \\
        -1 & 1 & 1
    \end{bmatrix}\hspace{0.5cm}
    U_5 = \begin{bmatrix}
        0  & 4 & 1 \\
        -1 & 1 & 0
    \end{bmatrix}\hspace{0.5cm}
    U_6 = \begin{bmatrix}
        1 & -3 & 2 \\
        0 & 0  & 0
    \end{bmatrix}
\]
\\
numa base ortogonal. \textcolor{COLOR2}{\textbf{Sugestão}}: Use o Python para se auxiliar nos cálculos matriciais necessários.
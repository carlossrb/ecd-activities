
\noindent \textcolor{COLOR1}{Questão 05)} Considere o espaço vetorial $M_{2\times3}\mathbb{R}$ com o produto interno $\langle U, V\rangle= tr(U^TV)$.
Aplique o processo de Gram-Schmidt para transformar a base
\\

\[
    U_1 = \begin{bmatrix}
        1 & 0 & 2 \\
        0 & 1 & 1
    \end{bmatrix}\hspace{0.5cm}
    U_2 = \begin{bmatrix}
        1 & 2 & 1 \\
        1 & 1 & 1
    \end{bmatrix}\hspace{0.5cm}
    U_3 = \begin{bmatrix}
        1 & 2 & 0 \\
        0 & 3 & 1
    \end{bmatrix}
\]
\[
    U_4 = \begin{bmatrix}
        1  & 0 & 1 \\
        -1 & 1 & 1
    \end{bmatrix}\hspace{0.5cm}
    U_5 = \begin{bmatrix}
        0  & 4 & 1 \\
        -1 & 1 & 0
    \end{bmatrix}\hspace{0.5cm}
    U_6 = \begin{bmatrix}
        1 & -3 & 2 \\
        0 & 0  & 0
    \end{bmatrix}
\]
\\
numa base ortogonal. \textcolor{COLOR2}{\textbf{Sugestão:}} Use o Python para se auxiliar nos cálculos matriciais necessários.
\\

O processo de Gram-Schmidt consiste no somatório a seguir para cada vetor da base:
\\

\begin{equation*}
    v_k = u_k - \sum_{i=1}^{k-1} \frac{\langle u_k,v_i\rangle}{\|v_i\|^2}v_i
\end{equation*}
\\

O codigo python para a transformação de base ortogonal é:\\

\begin{lstlisting}
import numpy as np
from fractions import Fraction
    
# Definindo impressao em formato fracionado
np.set_printoptions(formatter={"all": lambda o: str(Fraction(o)
.limit_denominator())})
    
    
# Calcula o traco
def trace(u, v):
    return np.trace(u.T @ v)
    
    
def proj(u, v):
    return trace(u, v) / trace(v, v)
    
    
# dic com valores de U
f = {
    "U1": np.array([[1, 0, 2], [0, 1, 1]]),
    "U2": np.array([[1, 2, 1], [1, 1, 1]]),
    "U3": np.array([[1, 2, 0], [0, 3, 1]]),
    "U4": np.array([[1, 0, 1], [-1, 1, 1]]),
    "U5": np.array([[0, 4, 1], [-1, 1, 0]]),
    "U6": np.array([[1, -3, 2], [0, 0, 0]])
}
# Calculos de V
V1 = f["U1"]
print(f"V1:\n{V1}")
    
V2 = f["U2"] - proj(f["U2"], V1) * V1
print(f"V2:\n{V2}")
    
V3 = f["U3"] - proj(f["U3"], V1) * V1 - proj(f["U3"], V2) * V2
print(f"V3:\n{V3}")
    
V4 = f["U4"] - proj(f["U4"], V1) * V1 - proj(f["U4"], V2) * V2 
- proj(f["U4"], V3) * V3
print(f"V4:\n{V4}")
    
V5 = f["U5"] - proj(f["U5"], V1) * V1 - proj(f["U5"], V2) * V2 
- proj(f["U5"], V3) * V3 - proj(f["U5"], V4) * V4
print(f"V5:\n{V5}")
    
V6 = f["U6"] - proj(f["U6"], V1) * V1 - proj(f["U6"], V2) * V2 
- proj(f["U6"], V3) * V3 - proj(f["U6"], V4) * V4 - proj(f["U6"], 
V5) * V5
print(f"V6:\n{V6}")
    
# V1:
# [[1 0 2]
#  [0 1 1]]
# V2:
# [[2/7 2 -3/7]
#  [1 2/7 2/7]]
# V3:
# [[0 0 -1]
#  [-1 2 0]]
# V4:
# [[6/19 4/19 -8/57]
#  [-32/57 -20/57 6/19]]
# V5:
# [[-6/5 6/5 6/5]
#  [-6/5 0 -6/5]]
# V6:
# [[1/2 0 0]
#  [0 0 -1/2]]
\end{lstlisting}
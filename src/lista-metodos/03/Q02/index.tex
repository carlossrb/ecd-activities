
\noindent \textcolor{COLOR1}{Questão 02)} Seja $V$ um espaço vetorial real e $v_1$, $v_2$, $v_3$, $v_4$ vetores de $V$ linearmente independentes. Mostre que $v_1$, $v_2-v_1$, $v_3-v_1$, $v_4-v_1$ são linearmente independentes
\\

Temos que em $V$ os vetores $v_1$, $v_2$, $v_3$, $v_4$ são LI. Para isso, há a condição de que nenhum deles pode ser nulo. Assim, $k_1v_1+ k_2v_2 + k_3v_3 + k_4v_4 = 0 $ com $k_1=k_2=k_3=k_4=0$.
\\

Para mostrarmos que a segunda relação entre os vetores é também LI, faz-se $k_1v_1+k_2(v_2-v_1)+k_3(v_3-v_1) + k_4(v_4-v_1)=0$. Ou ainda:
\\

\[
    (k_1-k_2-k_3-k_4)v_1 + k_2v_2 + k_3v_3 + k_4v_4 = 0
\]
\\

E sabendo que $k_2=k_3=k_4=0$, então:
\\

\[
    k_1-k_2-k_3-k_4\to k_1-0-0-0=0
\]
\\

O que implica em $k_1$ ser igual a zero também e atender a condição $k_1=k_2=k_3=k_4=0$, fazendo com que essa combinação de vetores seja linearmente independentes (LI).\\